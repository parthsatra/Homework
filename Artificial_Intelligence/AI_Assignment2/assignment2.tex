%%This is a very basic article template.
%%There is just one section and two subsections.
\documentclass[letterpaper]{article}
\usepackage[margin=1in,footskip=0.25in]{geometry}
% Package and macro definitions for CSC 503
% Originally prepared August 23, 2012 by Jon Doyle

%%% Page dimensions
\setlength{\oddsidemargin}{0in}
\setlength{\evensidemargin}{0in}
\setlength{\topmargin}{0in}
\setlength{\textheight}{9in}
\setlength{\textwidth}{6.5in}
\setlength{\headheight}{0in}
\setlength{\headsep}{0in}
\setlength{\footskip}{0.5in}

%%% Font and symbol definition packages
\usepackage{times} 
\usepackage{helvet} 
\usepackage{alltt}
\usepackage{amsfonts, amsmath}
\usepackage{amssymb}

%%% The modified Sellinger fitch.sty file
% These are Selinger's fitch.sty macros modified by Jon Doyle to
% conform with the nomenclature of inference rules in the Huth and
% Ryan textbook.  All the license and disclaimers of fitch.sty are
% maintained.  Extensions copyright (C) Jon Doyle, August 23, 2012.

% Macros for Fitch-style natural deduction. 
% Author: Peter Selinger, University of Ottawa
% Created: Jan 14, 2002
% Modified: Feb 8, 2005
% Version: 0.5
% Copyright: (C) 2002-2005 Peter Selinger
% Filename: fitch.sty
% Documentation: fitchdoc.tex
% URL: http://quasar.mathstat.uottawa.ca/~selinger/fitch/

% License:
%
% This program is free software; you can redistribute it and/or modify
% it under the terms of the GNU General Public License as published by
% the Free Software Foundation; either version 2, or (at your option)
% any later version.
%
% This program is distributed in the hope that it will be useful, but
% WITHOUT ANY WARRANTY; without even the implied warranty of
% MERCHANTABILITY or FITNESS FOR A PARTICULAR PURPOSE. See the GNU
% General Public License for more details.
%
% You should have received a copy of the GNU General Public License
% along with this program; if not, write to the Free Software Foundation, 
% Inc., 59 Temple Place, Suite 330, Boston, MA 02111-1307, USA.

% USAGE EXAMPLE:
% 
% The following is a simple example illustrating the usage of this
% package.  For detailed instructions and additional functionality, see
% the user guide, which can be found in the file fitchdoc.tex.
% 
% \[
% \begin{nd}
%   \hypo{1}  {P\vee Q}   
%   \hypo{2}  {\neg Q}                         
%   \open                              
%   \hypo{3a} {P}
%   \have{3b} {P}        \r{3a}
%   \close                   
%   \open
%   \hypo{4a} {Q}
%   \have{4b} {\neg Q}   \r{2}
%   \have{4c} {\bot}     \ne{4a,4b}
%   \have{4d} {P}        \be{4c}
%   \close                             
%   \have{5}  {P}        \oe{1,3a-3b,4a-4d}                 
% \end{nd}
% \]

{\chardef\x=\catcode`\*
\catcode`\*=11
\global\let\nd*astcode\x}
\catcode`\*=11

% References

\newcount\nd*ctr
\def\nd*render{\expandafter\ifx\expandafter\nd*x\nd*base\nd*x\the\nd*ctr\else\nd*base\ifnum\nd*ctr<0\the\nd*ctr\else\ifnum\nd*ctr>0+\the\nd*ctr\fi\fi\fi}
\expandafter\def\csname nd*-\endcsname{}

\def\nd*num#1{\nd*numo{\nd*render}{#1}\global\advance\nd*ctr1}
\def\nd*numopt#1#2{\nd*numo{$#1$}{#2}}
\def\nd*numo#1#2{\edef\x{#1}\mbox{$\x$}\expandafter\global\expandafter\let\csname nd*-#2\endcsname\x}
\def\nd*ref#1{\expandafter\let\expandafter\x\csname nd*-#1\endcsname\ifx\x\relax%
  \errmessage{Undefined natdeduction reference: #1}\else\mbox{$\x$}\fi}
\def\nd*noop{}
\def\nd*set#1#2{\ifx\relax#1\nd*noop\else\global\def\nd*base{#1}\fi\ifx\relax#2\relax\else\global\nd*ctr=#2\fi}
\def\nd*reset{\nd*set{}{1}}
\def\nd*refa#1{\nd*ref{#1}}
\def\nd*aux#1#2{\ifx#2-\nd*refa{#1}--\def\nd*c{\nd*aux{}}%
  \else\ifx#2,\nd*refa{#1}, \def\nd*c{\nd*aux{}}%
  \else\ifx#2;\nd*refa{#1}; \def\nd*c{\nd*aux{}}%
  \else\ifx#2.\nd*refa{#1}. \def\nd*c{\nd*aux{}}%
  \else\ifx#2)\nd*refa{#1})\def\nd*c{\nd*aux{}}%
  \else\ifx#2(\nd*refa{#1}(\def\nd*c{\nd*aux{}}%
  \else\ifx#2\nd*end\nd*refa{#1}\def\nd*c{}%
  \else\def\nd*c{\nd*aux{#1#2}}%
  \fi\fi\fi\fi\fi\fi\fi\nd*c}
\def\ndref#1{\nd*aux{}#1\nd*end}

% Layer A

% define various dimensions (explained in fitchdoc.tex):
\newlength{\nd*dim} 
\newdimen\nd*depthdim
\newdimen\nd*hsep
\newdimen\ndindent
\ndindent=1em
% user command to redefine dimensions
\def\nddim#1#2#3#4#5#6#7#8{\nd*depthdim=#3\relax\nd*hsep=#6\relax%
\def\nd*height{#1}\def\nd*thickness{#8}\def\nd*initheight{#2}%
\def\nd*indent{#5}\def\nd*labelsep{#4}\def\nd*justsep{#7}}
% set initial dimensions
\nddim{4.5ex}{3.5ex}{1.5ex}{1em}{1.6em}{.5em}{2.5em}{.2mm}

\def\nd*v{\rule[-\nd*depthdim]{\nd*thickness}{\nd*height}}
\def\nd*t{\rule[-\nd*depthdim]{0mm}{\nd*height}\rule[-\nd*depthdim]{\nd*thickness}{\nd*initheight}}
\def\nd*i{\hspace{\nd*indent}} 
\def\nd*s{\hspace{\nd*hsep}}
\def\nd*g#1{\nd*f{\makebox[\nd*indent][c]{$#1$}}}
\def\nd*f#1{\raisebox{0pt}[0pt][0pt]{$#1$}}
\def\nd*u#1{\makebox[0pt][l]{\settowidth{\nd*dim}{\nd*f{#1}}%
    \addtolength{\nd*dim}{2\nd*hsep}\hspace{-\nd*hsep}\rule[-\nd*depthdim]{\nd*dim}{\nd*thickness}}\nd*f{#1}}

% Lists

\def\nd*push#1#2{\expandafter\gdef\expandafter#1\expandafter%
  {\expandafter\nd*cons\expandafter{#1}{#2}}}
\def\nd*pop#1{{\def\nd*nil{\gdef#1{\nd*nil}}\def\nd*cons##1##2%
    {\gdef#1{##1}}#1}}
\def\nd*iter#1#2{{\def\nd*nil{}\def\nd*cons##1##2{##1#2{##2}}#1}}
\def\nd*modify#1#2#3{{\def\nd*nil{\gdef#1{\nd*nil}}\def\nd*cons##1##2%
    {\advance#2-1 ##1\advance#2 1 \ifnum#2=1\nd*push#1{#3}\else%
      \nd*push#1{##2}\fi}#1}}

\def\nd*cont#1{{\def\nd*t{\nd*v}\def\nd*v{\nd*v}\def\nd*g##1{\nd*i}%
    \def\nd*i{\nd*i}\def\nd*nil{\gdef#1{\nd*nil}}\def\nd*cons##1##2%
    {##1\expandafter\nd*push\expandafter#1\expandafter{##2}}#1}}

% Layer B

\newcount\nd*n
\def\nd*beginb{\begingroup\nd*reset\gdef\nd*stack{\nd*nil}\nd*push\nd*stack{\nd*t}%
  \begin{array}{l@{\hspace{\nd*labelsep}}l@{\hspace{\nd*justsep}}l}}
\def\nd*resumeb{\begingroup\begin{array}{l@{\hspace{\nd*labelsep}}l@{\hspace{\nd*justsep}}l}}
\def\nd*endb{\end{array}\endgroup}
\def\nd*hypob#1#2{\nd*f{\nd*num{#1}}&\nd*iter\nd*stack\relax\nd*cont\nd*stack\nd*s\nd*u{#2}&}
\def\nd*haveb#1#2{\nd*f{\nd*num{#1}}&\nd*iter\nd*stack\relax\nd*cont\nd*stack\nd*s\nd*f{#2}&}
\def\nd*havecontb#1#2{&\nd*iter\nd*stack\relax\nd*cont\nd*stack\nd*s\nd*f{\hspace{\ndindent}#2}&}
\def\nd*hypocontb#1#2{&\nd*iter\nd*stack\relax\nd*cont\nd*stack\nd*s\nd*u{\hspace{\ndindent}#2}&}

\def\nd*openb{\nd*push\nd*stack{\nd*i}\nd*push\nd*stack{\nd*t}}
\def\nd*closeb{\nd*pop\nd*stack\nd*pop\nd*stack}
\def\nd*guardb#1#2{\nd*n=#1\multiply\nd*n by 2 \nd*modify\nd*stack\nd*n{\nd*g{#2}}}

% Layer C

\def\nd*clr{\gdef\nd*cmd{}\gdef\nd*typ{\relax}}
\def\nd*sto#1#2#3{\gdef\nd*typ{#1}\gdef\nd*byt{}%
  \gdef\nd*cmd{\nd*typ{#2}{#3}\nd*byt\\}}
\def\nd*chtyp{\expandafter\ifx\nd*typ\nd*hypocontb\def\nd*typ{\nd*havecontb}\else\def\nd*typ{\nd*haveb}\fi}
\def\nd*hypoc#1#2{\nd*chtyp\nd*cmd\nd*sto{\nd*hypob}{#1}{#2}}
\def\nd*havec#1#2{\nd*cmd\nd*sto{\nd*haveb}{#1}{#2}}
\def\nd*hypocontc#1{\nd*chtyp\nd*cmd\nd*sto{\nd*hypocontb}{}{#1}}
\def\nd*havecontc#1{\nd*cmd\nd*sto{\nd*havecontb}{}{#1}}
\def\nd*by#1#2{\ifx\nd*x#2\nd*x\gdef\nd*byt{\mbox{#1}}\else\gdef\nd*byt{\mbox{#1, \ndref{#2}}}\fi}

% multi-line macros
\def\nd*mhypoc#1#2{\nd*mhypocA{#1}#2\\\nd*stop\\}
\def\nd*mhypocA#1#2\\{\nd*hypoc{#1}{#2}\nd*mhypocB}
\def\nd*mhypocB#1\\{\ifx\nd*stop#1\else\nd*hypocontc{#1}\expandafter\nd*mhypocB\fi}
\def\nd*mhavec#1#2{\nd*mhavecA{#1}#2\\\nd*stop\\}
\def\nd*mhavecA#1#2\\{\nd*havec{#1}{#2}\nd*mhavecB}
\def\nd*mhavecB#1\\{\ifx\nd*stop#1\else\nd*havecontc{#1}\expandafter\nd*mhavecB\fi}
\def\nd*mhypocontc#1{\nd*mhypocB#1\\\nd*stop\\}
\def\nd*mhavecontc#1{\nd*mhavecB#1\\\nd*stop\\}

\def\nd*beginc{\nd*beginb\nd*clr}
\def\nd*resumec{\nd*resumeb\nd*clr}
\def\nd*endc{\nd*cmd\nd*endb}
\def\nd*openc{\nd*cmd\nd*clr\nd*openb}
\def\nd*closec{\nd*cmd\nd*clr\nd*closeb}
\let\nd*guardc\nd*guardb

% Layer D

% macros with optional arguments spelled-out
\def\nd*hypod[#1][#2]#3[#4]#5{\ifx\relax#4\relax\else\nd*guardb{1}{#4}\fi\nd*mhypoc{#3}{#5}\nd*set{#1}{#2}}
\def\nd*haved[#1][#2]#3[#4]#5{\ifx\relax#4\relax\else\nd*guardb{1}{#4}\fi\nd*mhavec{#3}{#5}\nd*set{#1}{#2}}
\def\nd*havecont#1{\nd*mhavecontc{#1}}
\def\nd*hypocont#1{\nd*mhypocontc{#1}}
\def\nd*base{undefined}
\def\nd*opend[#1]#2{\nd*cmd\nd*clr\nd*openb\nd*guard{#1}#2}
\def\nd*close{\nd*cmd\nd*clr\nd*closeb}
\def\nd*guardd[#1]#2{\nd*guardb{#1}{#2}}

% Handling of optional arguments.

\def\nd*optarg#1#2#3{\ifx[#3\def\nd*c{#2#3}\else\def\nd*c{#2[#1]{#3}}\fi\nd*c}
\def\nd*optargg#1#2#3{\ifx[#3\def\nd*c{#1#3}\else\def\nd*c{#2{#3}}\fi\nd*c}

\def\nd*five#1{\nd*optargg{\nd*four{#1}}{\nd*two{#1}}}
\def\nd*four#1[#2]{\nd*optarg{0}{\nd*three{#1}[#2]}}
\def\nd*three#1[#2][#3]#4{\nd*optarg{}{#1[#2][#3]{#4}}}
\def\nd*two#1{\nd*three{#1}[\relax][]}

\def\nd*have{\nd*five{\nd*haved}}
\def\nd*hypo{\nd*five{\nd*hypod}}
\def\nd*open{\nd*optarg{}{\nd*opend}}
\def\nd*guard{\nd*optarg{1}{\nd*guardd}}

\def\nd*init{%
  \let\open\nd*open%
  \let\close\nd*close%
  \let\hypo\nd*hypo%
  \let\have\nd*have%
  \let\hypocont\nd*hypocont%
  \let\havecont\nd*havecont%
  \let\by\nd*by%
  \let\guard\nd*guard%
  \def\ii{\by{$\rightarrow$i}}%    %JD modification
  \def\ie{\by{$\rightarrow$e}}%    %JD modification
  \def\bi{\by{$\leftrightarrow$i}}%    %JD modification
  \def\be{\by{$\leftrightarrow$e}}%    %JD modification
  \def\Ai{\by{$\forall$i}}%
  \def\Ae{\by{$\forall$e}}%
  \def\Ei{\by{$\exists$i}}%
  \def\Ee{\by{$\exists$e}}%
  \def\ai{\by{$\wedge$i}}%
  \def\ae{\by{$\wedge$e}}%
  \def\aeone{\by{$\wedge$e$_1$}}%    %JD addition
  \def\aetwo{\by{$\wedge$e$_2$}}%    %JD addition
  \def\ai{\by{$\wedge$i}}%
  \def\aione{\by{$\wedge$i$_1$}}%    %JD addition for false proof questions
  \def\aitwo{\by{$\wedge$i$_2$}}%    %JD addition for false proof questions
  \def\ae{\by{$\wedge$e}}%
  \def\oi{\by{$\vee$i}}%
  \def\oione{\by{$\vee$i$_1$}}%    %JD addition
  \def\oitwo{\by{$\vee$i$_2$}}%    %JD addition
  \def\oe{\by{$\vee$e}}%
  \def\oeone{\by{$\vee$e$_1$}}%    %JD addition for false proof questions
  \def\oetwo{\by{$\vee$e$_2$}}%    %JD addition for false proof questions
  \def\ni{\by{$\neg$i}}%
  \def\ne{\by{$\neg$e}}%
  \def\be{\by{$\bot$e}}%
  \def\nne{\by{$\neg\neg$e}}%JD addition
  \def\nni{\by{$\neg\neg$i}}%JD addition
  \def\r{\by{R}}%
  \def\copy{\by{copy}}% %JD addition
  \def\pbc{\by{PBC}}%    %JD addition
  \def\mt{\by{MT}}%    %JD addition
  \def\lem{\by{LEM}}%    %JD addition
  \def\bn{\by{$\bot_\neg$}}%
  \def\tn{\by{$\top_\neg$}}%
  \def\premise{\by{premise}} % JD addition
  \def\assumption{\by{assumption}} % JD addition
  \def\Implied{\by{Implied}} % JD addition
  \def\Assignment{\by{Assignment}} % JD addition
  \def\IfStatement{\by{If-Statement}} % JD addition
  \def\PartialWhile{\by{Partial-While}} % JD addition
  \def\InvariantGuard{\by{Invariant Hyp. and Guard}} % JD addition
  \def\TotalWhile{\by{Total-While}} % JD addition
  \def\Ki{\by{$K$i}}%
  \def\Ke{\by{$K$e}}%
  \def\KT{\by{$KT$}}%
  \def\Kfour{\by{$K4$}}%
  \def\Kfive{\by{$K5$}}%
  \def\EKi{\by{$E$i}}%
  \def\EKe{\by{$E$e}}%
  \def\CKi{\by{$C$i}}%
  \def\CKe{\by{$C$e}}%
  \def\CK{\by{$CK$}}%
  \def\Cfour{\by{$C4$}}%
  \def\Cfive{\by{$C5$}}%
}

\newenvironment{nd}{\begingroup\nd*init\nd*beginc}{\nd*endc\endgroup}
\newenvironment{ndresume}{\begingroup\nd*init\nd*resumec}{\nd*endc\endgroup}

\catcode`\*=\nd*astcode

% End of file fitch.sty



\newcommand{\Z}{\mathbb{Z}}
\newcommand{\Q}{\mathbb{Q}}
\newcommand{\R}{\mathbb{R}}
\newcommand{\N}{\mathbb{N}}
\def\land{\wedge}
\def\lor{\vee}
\def\implies{\rightarrow}
\def\iff{\leftrightarrow}
\def\turn{\vdash}
\def\Cn{\text{Cn}}
\def\Th{\text{Th}}
\def\defeq{\stackrel{\rm def}{=}}

%%% The environment for providing answers to problems
\newenvironment{answer}%
{\par\noindent\textbf{Answer}\par\noindent}%
{}


\title{CSC 520 Homework Assignment 2}
\author{Parth Satra - 200062999}

\begin{document}
\maketitle

\begin{enumerate}
  \item (50 Points) This question concerns route-finding, with comparison
  of several search algorithms. This time, we're in the U.S. The solution consists
  of the series of cities a network packet must pass through, 
  each city connected to one or more others by network links of the indicated length. 
  There are no other network links.
  
  Now perform some experiments.
  	\begin{enumerate}
  	  \item (10 points) Experiment with executing your implementation of
  	  A* to find various paths, until you understand the meaning of the output. 
  	  Are there any pairs of cities (A,B) for which the algorithm finds a different path from B to A than from A to B? 
  	  Are there any pairs of cities (A,B) for which the algorithm expands a different total number of nodes from B to A 
  	  than from A to B?
  	  Output in such cases should consist of the following:
  	  \begin{answer}
  	  	A* always finds the same path from A to B and B to A. This is because A*
  	  	takes into account both the actual distance and heuristics to calculate
  	  	the best possible path. So if while going from A to B if C is the city in
  	  	best possible path then from B to A also C will be the present in the best
  	  	possible path. But the expanded nodes may vary since A* will now explore
  	  	nodes in reverse direction and depending on the geography different nodes
  	  	will be expanded. Example for both the cases is as follows:
		\bigskip
		
  	  	\textbf{Sacramento to Portland: A*}
		\begin{enumerate}
			\item A comma separated list of expanded nodes (the closed list) :
			sacramento, reno, stockton, pointReyes, modesto, sanFrancisco, oakland,
			redding, medford, sanJose, eugene, salem
			\item The number of nodes expanded : 
			12
			\item A comma-separated list of nodes in the solution path :
			sacramento, pointReyes, redding, medford, eugene, salem, portland
			\item The number of nodes in the path : 
			7
			\item The total distance from A to B in the solution path.
			755
		\end{enumerate}
		\bigskip
		
		\textbf{Portland to Sacramento: A*}
		\begin{enumerate}
			\item A comma separated list of expanded nodes (the closed list) : 
			portland, salem, eugene, medford, redding, pointReyes
			\item The number of nodes expanded : 
			6
			\item A comma-separated list of nodes in the solution path :
			portland, salem, eugene, medford, redding, pointReyes, sacramento
			\item The number of nodes in the path : 
			7
			\item The total distance from A to B in the solution path : 
			755
		\end{enumerate}
	  \end{answer}
	  \bigskip
	  
	  \item (10 Points) Change your code so as to implement greedy search, as
	  discussed in the web notes.
	  \begin{answer}
	  	Code present in SeachUSA.java and searchType is greedy.
	  \end{answer}
	  \bigskip
	  
	  \item (10 Points) Do enough exploration to find at least one path that is
	  longer using greedy search than that found using A*, 
	  or to satisfy yourself that there are no such paths. 
	  Find at least one path that is found by expanding more nodes than the comparable path using A*, 
	  or satisfy yourself that there are no such paths. 
	  If there is such a path, list the nodes in the path and the total distance.
	  \begin{answer}
	  	The below examples shows both the instances. A* can be seen to have shorter
	  	path and also expanded less number of nodes than Greedy. 
	  	\bigskip
	  	
	  	\textbf{Los Angeles to Sacramento: A*}
		\begin{enumerate}
			\item A comma separated list of expanded nodes (the closed list) : 
			losAngeles, bakersfield, fresno, modesto, stockton
			\item The number of nodes expanded : 
			5
			\item A comma-separated list of nodes in the solution path :
			losAngeles, bakersfield, fresno, modesto, stockton, sacramento
			\item The number of nodes in the path : 
			6
			\item The total distance from A to B in the solution path : 
			408
		\end{enumerate}
		\bigskip
		
		\textbf{Los Angeles to Sacramento: Greedy}
		\begin{enumerate}
			\item A comma separated list of expanded nodes (the closed list) : 
			losAngeles, sanLuisObispo, salinas, sanJose, oakland, sanFrancisco
			\item The number of nodes expanded : 
			6
			\item A comma-separated list of nodes in the solution path :
			losAngeles, sanLuisObispo, salinas, sanJose, oakland, sanFrancisco,
			sacramento
			\item The number of nodes in the path : 
			7
			\item The total distance from A to B in the solution path : 
			495
		\end{enumerate}
	  \end{answer}
	  \bigskip
	  
	  \item (10 Points) Change your code so as to implement uniform cost search, as
	  discussed in the web notes.
	  \begin{answer}
		The code is present in SearchUSA.java file and the searchType for Uniform
		Cost search is 'uniform'.
	  \end{answer}	
		\bigskip
		
	  \item (10 Points) Do enough exploration to find at least one path that is
	  longer using uniform cost than that found using A*, or to satisfy yourself that there are no such paths. 
	  Find at least one path that is found by expanding more nodes than the comparable path using A*, 
	  or satisfy yourself that there are no such paths. 
	  If there is such a path, list the nodes in the path and the total distance.
	  \begin{answer}
	  	
	  	The below example shows that the uniform cost expands more nodes than A*
	  	does as A* is guided by a heuristic function in the right direction. But
	  	eventually Uniform cost also finds the same solution path as that of the A*
	  	since it also explores the shortest path. 
	  	\bigskip
	  	
	  	\textbf{Los Angeles to Sacramento: A*}
		\begin{enumerate}
			\item A comma separated list of expanded nodes (the closed list) : 
			losAngeles, bakersfield, fresno, modesto, stockton
			\item The number of nodes expanded : 
			5
			\item A comma-separated list of nodes in the solution path :
			losAngeles, bakersfield, fresno, modesto, stockton, sacramento
			\item The number of nodes in the path : 
			6
			\item The total distance from A to B in the solution path : 
			408
		\end{enumerate}
		\bigskip
		
		\textbf{Los Angeles to Sacramento: Uniform Cost Search}
		\begin{enumerate}
			\item A comma separated list of expanded nodes (the closed list) : 
			losAngeles, bakersfield, sanDiego, sanLuisObispo, fresno, lasVegas, 
			yuma, salinas, modesto, sanJose, stockton, oakland, sanFrancisco
			\item The number of nodes expanded : 
			13
			\item A comma-separated list of nodes in the solution path :
			losAngeles,bakersfield,fresno,modesto,stockton,sacramento
			\item The number of nodes in the path : 
			6
			\item The total distance from A to B in the solution path : 
			408
		\end{enumerate}		
	  \end{answer}
	  \bigskip
	  
	  \item As part of your answer, compare the solution paths and explain what
	  happened, especially any weird behavior you might detect.
	  \begin{answer}
	  	In all of the above examples we see different paths or nodes expanded as
	  	there are multiple maths to the solution. The main differences in all the
	  	three discussed algorithms is 
	  	
	  	Uniform cost search tries to explore all the shortest path without any
	  	additional knowledge about the state space. Hence it explores many short
	  	paths before exploring the optimal path which might not be the shortest from
	  	the source. Hence uniform cost might end up expanding lots of nodes in
	  	search of optimal path.
	  	
	  	Greedy search tries to find the state closest to the goal without
	  	considering the cost from the source which might turn out to be expensive.
	  	Hence greedy search doesnt always give the optimal solution. 
	  	
	  	A* search tries to search for the optimal cost using a heuristic function
	  	which provides it a sense of direction to the goal state. Having said that
	  	A* search can also explore states which are not on the optimal path. 
	  \end{answer}
  	\end{enumerate}
  	\bigskip
  	
  	\item (20 Points) Tic-tac-toe (also known as Noughts and Crosses) is a
  	two-person, zero-sum game, in which player X and player O alternate placing their 
  	symbols in one of the blank spaces in a 3-by-3 grid.
  	
  	The first player to place three of his symbols in a row -- horizontally,
  	vertically, or diagonally -- wins.
	
	\begin{enumerate}

	  \item Beginning from the position specified with X's turn to move, construct by hand
	   the game-tree for the rest of the game. Assume the search horizon is the end of the
	   game on all branches. Hint: You can save a lot of work by taking advantage
	   of symmetry
	   \begin{answer}
	   		\textbf{The answer to all sub-questions is available in the $min\_max.svg$
	   		or $min\_max.jpg$. Please open the svg file in Google Chrome.}
	   \end{answer}
	   
	  \item (5 points) Suppose the static evaluation function scores a win for O as
	  +1, a draw as 0, and a loss for O as -1. At each level of your sketch of the game tree, 
	  indicate the value of each node based on its children. Indicate the best next move for X.

	  \item (5 points) Now use α-β pruning. At each level of your sketch of the
	  game tree, indicate the value of the nodes and indicate any nodes that would be pruned.
	  Explicitly indicate the final value of the α-β interval is for each node. Indicate the best next move for X.
	\end{enumerate}
	
	\newpage
	\item (12 points) Use Propositional Logic to determine whether or not the
	following set of requirements is logically consistent. In other words, represent the following sentences 
	in Propositional Logic, convert to Conjunctive Normal Form, and run Resolution until a contradiction is derived,
	or else show a model of all the expressions showing that no contradiction exists.
	
	The system is in multiuser state if and only if it is operating normally. 
	If the system is operating normally, the kernel is functioning. 
	Either the kernel is not functioning or the system is in interrupt mode. 
	If the system is not in multiuser state, then it is in interrupt mode. 
	The system is not in interrupt mode.
	Use the following lexicon:
	Propositional symbols:
	
	m -- The system is in multiuser state.
	
	n -- The system is operating normally.
	
	k -- The kernel is functioning.
	
	i -- The system is in interrupt mode.
	
	\begin{answer}
		The system is in multiuser state if and only if it is operating normally : $m
		\leftrightarrow n$
		
		If the system is operating normally, the kernel is functioning: $n \implies k$
		
		Either the kernel is not functioning or the system is in interrupt mode:
		$\neg k \lor i$
		
		If the system is not in multiuser state, then it is in interrupt mode : $\neg
		m \implies i$
		
		The system is not in interrupt mode : $\neg i$
		\bigskip
		
		\textbf{Conversion to CNF and applying Resolution}
		\bigskip
		
		\begin{enumerate}
		  \item $\neg m \lor n$
		  \item $\neg n \lor m$
		  \item $\neg n \lor k$
		  \item $\neg k \lor i$
		  \item $m \lor i$
		  \item $\neg i$
		  \item $\neg k $ (From : f + d)
		  \item $\neg n$ (From : f + c)
		  \item $\neg m$ (From : h + a)
		  \item $\neg n$ (From : i + b)
		  \item $\neg m$ (From : j + a)
		  \item $i$ (From : k + e) Not taking step b to resolve as it leads to a loop.
		  \item $\square$ (From : l + f)
		  
		\end{enumerate}
		\bigskip
		
		\textbf{Thus the above resolution shows that the system is inconsistent and
		contradictory.}
		\bigskip
	\end{answer}
	
	\newpage
	\item (18 points) Consider the following English sentences.


	All cats are mammals.

	The head of a cat is the head of a mammal.

	\begin{enumerate}
		\item (6 points) Convert the sentences into first-order predicate logic. Be
		extremely careful about quantification; because not everything in the 
		universe is a mammal, or a cat, or a head, you will need both kinds of 
		quantification. 
		Use the following lexicon:
	 	
	 	Predicates: 
	 	
	 	cat(X) -- X is a cat.
        
        mammal(X) -- X is a mammal.
        
        headOf(H,X) -- H is the head of X.
        
        \begin{answer}
        	
        	The sentenses in First order Logic can be represented as 
			\bigskip
			        	
        	First formula: 
        	$\forall X (cat(X) \implies mammal(X))$
        	
        	Second Formula: 
        	$\forall H (\exists X (cat(X) \land headOf(H,X)) \implies \exists Y
        	(mammal(Y) \land headOf(H, Y)))$
        	
        	\bigskip
        	
        \end{answer}
        
        \item (4 Points) Convert the logic statements into CNF. HINT: With this
        lexicon, you will need two Skolem constants if you're doing this correctly.
        \begin{answer}
        	
        	CNF Conversion
        	\bigskip
        	
        	\textbf{For the first formula:} 
        	
        	$\forall X (cat(X) \implies mammal(X))$
        	
        	$\forall X (\neg cat(X) \lor mammal(X))$
        	\bigskip
        	
        	\textbf{CNF : $\neg cat(X) \lor mammal(X)$}
        	
        	\bigskip
        	
        	\textbf{For the second formula.}
        	
        	$\forall H (\exists X (cat(X) \land headOf(H,X)) \implies \exists Y
        	(mammal(Y) \land headOf(H, Y)))$
        	
        	$\forall H (\neg (\exists X (cat(X) \land headOf(H,X))) \lor
        	\exists Y (mammal(Y) \land headOf(H, Y)))$
			
			$\forall H (\forall X (\neg cat(X) \lor \neg headOf(H,X)) \lor
        	\exists Y (mammal(Y) \land headOf(H, Y)))$
        	
        	$\forall H \forall X \exists Y((\neg cat(X) \lor \neg headOf(H,X))
        	\lor (mammal(Y) \land headOf(H, Y)))$
        	
        	$(\neg cat(X) \lor \neg headOf(H,X)) \lor (mammal(F(X, H))
        	\land headOf(H, F(X, H)))$
        	
        	$(\neg cat(X) \lor \neg headOf(H,X)) \lor (mammal(F(X, H)) \land
        	headOf(H, F(X, H)))$
        	\bigskip

			\textbf{CNF: $\neg cat(X_1) \lor \neg headOf(H_!,X_1) \lor mammal(F(X_1,
			H_1))$}
        	
        	\textbf{CNF: $\neg cat(X_2) \lor \neg headOf(H_2,X_2) \lor headOf(H_2,
        	F(X_2, H_2))$}
        	        	
        \end{answer}
        \bigskip 
        
        \item (6 Points) Using resolution and the 4-part heuristic presented in
        class, prove using FOPL resolution that the second of the original 
        sentences follows from the first. 
        Number your clauses, and indicate explicitly step-by-step what resolves together, 
        under what substitution.
        \begin{answer}
        	\textbf{CNF for negation of the second formula}
        	\bigskip
        	
        	$\neg (\forall H (\exists X (cat(X) \land headOf(H,X)) \implies \exists
        	Y (mammal(Y) \land headOf(H, Y))))$
        	
        	$\neg (\forall H \neg (\exists X (cat(X) \land headOf(H,X)) \lor
        	(\exists Y (mammal(Y) \land headOf(H, Y)))))$
        	
        	$\exists H (\exists X (cat(X) \land headOf(H,X)) \land
        	\neg (\exists Y (mammal(Y) \land headOf(H, Y))))$
        	
        	$\exists H (\exists X (cat(X) \land headOf(H,X)) \land
        	(\forall Y \neg (mammal(Y) \lor \neg headOf(H, Y))))$
        	
        	$\exists H \exists X \forall Y ((cat(X) \land headOf(H,X)) \land
        	(\neg (mammal(Y) \lor \neg headOf(H, Y))))$
        	
        	$((cat(X_0) \land headOf(H_0,X_0)) \land (\neg (mammal(Y) \lor \neg
        	headOf(H_0, Y))))$
        	\bigskip
        	
        	CNF : 
        	
        	$cat(X_0)$
        	
        	$headOf(H_0, X_0)$
        	
        	$\neg mammal(Y) \lor \neg headOf(H_0, Y)$	
        	
        	\bigskip
        	
        	\textbf{Resolution Proof:}
        	
        	\begin{enumerate}
        	  \item $\neg cat(X) \lor mammal(X)$
        	  \item $cat(X_0)$
        	  \item $headOf(H_0, X_0)$
        	  \item $\neg mammal(Y) \lor \neg headOf(H_0, Y)$	
        	  \item $\neg cat(X) \lor \neg headOF(H_0, X)$ (From iv + i)
        	  (substituting X/Y)
        	  \item $\neg headOF(H_0, X_0)$ (From v + ii) (Substitution : $X_0/X$)
        	  \item $\square$ (From vi + iii) (Substitution : empty)
        	\end{enumerate}
        	\bigskip
        	
        	\textbf{Thus we can derive the conclusion which is the second statement by
        	substitutions ($X/Y$, $X_0/X$, $()$).}
        	
        \end{answer}
        \bigskip
        
        \item (2 Points) Show a shorter proof that doesn't use the heuristic, if
        you can find one.
        \begin{answer}
        	The proof above is the shortest possible one.
        \end{answer}
        
	\end{enumerate}
\end{enumerate}

\end{document}
