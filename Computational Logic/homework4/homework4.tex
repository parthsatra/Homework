\documentclass{article}
% Package and macro definitions for CSC 503
% Originally prepared August 23, 2012 by Jon Doyle

%%% Page dimensions
\setlength{\oddsidemargin}{0in}
\setlength{\evensidemargin}{0in}
\setlength{\topmargin}{0in}
\setlength{\textheight}{9in}
\setlength{\textwidth}{6.5in}
\setlength{\headheight}{0in}
\setlength{\headsep}{0in}
\setlength{\footskip}{0.5in}

%%% Font and symbol definition packages
\usepackage{times} 
\usepackage{helvet} 
\usepackage{alltt}
\usepackage{amsfonts, amsmath}
\usepackage{amssymb}

%%% The modified Sellinger fitch.sty file
\input{fitchhr.sty}

\newcommand{\Z}{\mathbb{Z}}
\newcommand{\Q}{\mathbb{Q}}
\newcommand{\R}{\mathbb{R}}
\newcommand{\N}{\mathbb{N}}
\def\land{\wedge}
\def\lor{\vee}
\def\implies{\rightarrow}
\def\iff{\leftrightarrow}
\def\turn{\vdash}
\def\Cn{\text{Cn}}
\def\Th{\text{Th}}
\def\defeq{\stackrel{\rm def}{=}}

%%% The environment for providing answers to problems
\newenvironment{answer}%
{\par\noindent\textbf{Answer}\par\noindent}%
{}


\title{CSC 503 Homework Assignment 4}
\author{Due September 17, 2014}
\date{September 10, 2014}

\begin{document}
\maketitle


In using the Fitch macros to typeset proofs in first order logic, one
introduces a dummy variable $x$ by means of the command
\verb+\open[x]+. 

Unless directed otherwise, follow the convention of the text and
assume that $a,b,c,d,e$ are constant symbols, $f,g,h$ are function
symbols, and $u,v,x,y,z$ are variable symbols.


\begin{enumerate}

\item Let $c$ and $d$ be constants, $f$ a function symbol with one
  argument, $g$ a function symbol with two arguments, $h$ a
  function symbol with three arguments, and $P$ and $Q$ predicate
  symbols with three arguments.  Indicate, for each of the following
  strings, which strings are formulas in predicate logic, and specify
  a reason for failure for strings which are not.
  \begin{enumerate}
  \item {[5 points]} $\forall x P(f(d),h(g(c,x),d,y),x)$
  \begin{answer}
  	This formula is a \textbf{valid} formula in predicate logic
  \end{answer}
  \item {[5 points]} $\forall x P(f(d),h(P(x,y,d),d,y),x)$
  \begin{answer}
  	This formula is \textbf{invalid} as $P(x,y,d)$ is not a term and all the
  	paramters to the function must be a term.
  \end{answer}
  \item {[5 points]} $\forall x (Q(z,z,z) \implies P(z))$
  \begin{answer}
  	This formula is \textbf{invalid} formula in predicate logic since the
  	predicate $P$ is incorrectly used with only one argument.
  \end{answer}
  \item {[5 points]} $\forall x \forall y (g(x,y) \implies P(x,y,x))$
  \begin{answer}
  	This formula is \textbf{invalid} formula as $g(x,y)$ is a term and is not a
  	formula. Hence $\phi_! \implies \phi_2$ is a formula only when $\phi_1$ and
  	$\phi_2$ are formulas.
  \end{answer}
  \item {[5 points]} $Q(c,d,c)$
  \begin{answer}
  	This formula is a \textbf{valid} formula in predicate logic
  \end{answer}
  \item {[5 points]} $\forall x \forall y P(x,x,x)$
  \begin{answer}
  	This formula is a \textbf{valid} formula in predicate logic
  \end{answer}
  \end{enumerate}

\item Let $P$ be a predicate symbol with arity 2, and let $\phi$ be
  the formula
  \begin{displaymath}
    \exists x (P(y,z) \land (\forall y (\neg P(y,x) \lor P(y,z))))
  \end{displaymath}
  \begin{enumerate}
  \item {[5 points]} Indicate, for each occurrence of each variable in
    $\phi$, whether that occurrence is free or bound.
    \begin{answer}
    	The variables in $\phi$ include $x, y$ and $z$.
    	The highlighted variables in $\phi$ are free.
    	
    	\begin{displaymath}
    		\exists x (P(\textbf{y},\textbf{z}) \land (\forall y (\neg P(y,x) \lor
    		P(y,\textbf{z}))))
  		\end{displaymath}
  		
    \end{answer}
  \item {[5 points]} List all variables which occur free and bound in
    $\phi$.
    \begin{answer}
    	The variables in $\phi$ include $x, y$ and $z$.
    	The highlighted variables in $\phi$ are free and the rest of the occurences
    	are bounded with $x$ bounded by $\exists x$ and $y$ on the right hand side
    	by $\forall y$
    	
    	\begin{displaymath}
    		\exists x (P(\textbf{y},\textbf{z}) \land (\forall y (\neg P(y,x) \lor
    		P(y,\textbf{z}))))
  		\end{displaymath}
    \end{answer}
  \item {[10 points]} Compute $\phi[t/x]$ for $t = g(f(g(y,y)),y)$.  Is
    $t$ free for $x$ in $\phi$?
    \begin{answer}
    	$t$ is not free for $x$ in $\phi$ as there is no free $x$ in $\phi$ to be
    	replaced by $t$. Thus $\phi[t/x]$ will remain $\phi$. 
    \end{answer}
  \item {[10 points]} Compute $\phi[t/y]$ for $t = g(f(g(y,y)),y)$ Is
    $t$ free for $y$ in $\phi$?
    \begin{answer}
    	$t$ is free for $y$ in $\phi$ as there a free instance of $y$ which on
    	replacement with $t$ does not bound it. 
    	Thus we have,
    	\begin{displaymath}
 		    \exists x (P(\textbf{g(f(g(y,y)),y)},z) \land (\forall y (\neg P(y,x)
 		    \lor P(y,z))))
 		\end{displaymath}
    \end{answer}
  \item {[10 points]} Compute $\phi[t/z]$ for $t = g(f(g(y,y)),y)$ Is
    $t$ free for $z$ in $\phi$?
    \begin{answer}
    	$t$ is not free for $z$ in $\phi$ as when we replace $t$ for free
    	instances of $z$ in $\phi$ we add additional bounding condition to the
    	variable $y$ in $t$. 
    \end{answer}
  \end{enumerate}

\item {[30 points]} Find a proof for $\forall x (P(x) \land Q(x))
  \turn \forall x ( P(x) \implies Q(x))$.
  \begin{answer}
  	\[
  		\begin{nd}
  			\hypo{1} {\forall x (P(x) \land Q(x))} \premise{}
  			\open[x_0]
  				\have{2a} {P(x_0) \land Q(x_0)} \Ae {1}
  				\open 
  					\hypo{2bi} {P(x_0)} \assumption{}
  					\have{2bii} {Q(x_0)} \aetwo {2a}
  				\close
				\have{2c} {P(x_0) \implies Q(x_0)} \ii {2bi-2bii}
  			\close 
  			\have{3} {\forall x (P(x) \implies Q(x))} \Ai {2a-2c}
  		\end{nd}
  	\]	
  	
  \end{answer}

\end{enumerate}
\end{document}
