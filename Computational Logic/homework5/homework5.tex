\documentclass{article}
% Package and macro definitions for CSC 503
% Originally prepared August 23, 2012 by Jon Doyle

%%% Page dimensions
\setlength{\oddsidemargin}{0in}
\setlength{\evensidemargin}{0in}
\setlength{\topmargin}{0in}
\setlength{\textheight}{9in}
\setlength{\textwidth}{6.5in}
\setlength{\headheight}{0in}
\setlength{\headsep}{0in}
\setlength{\footskip}{0.5in}

%%% Font and symbol definition packages
\usepackage{times} 
\usepackage{helvet} 
\usepackage{alltt}
\usepackage{amsfonts, amsmath}
\usepackage{amssymb}

%%% The modified Sellinger fitch.sty file
\input{fitchhr.sty}

\newcommand{\Z}{\mathbb{Z}}
\newcommand{\Q}{\mathbb{Q}}
\newcommand{\R}{\mathbb{R}}
\newcommand{\N}{\mathbb{N}}
\def\land{\wedge}
\def\lor{\vee}
\def\implies{\rightarrow}
\def\iff{\leftrightarrow}
\def\turn{\vdash}
\def\Cn{\text{Cn}}
\def\Th{\text{Th}}
\def\defeq{\stackrel{\rm def}{=}}

%%% The environment for providing answers to problems
\newenvironment{answer}%
{\par\noindent\textbf{Answer}\par\noindent}%
{}


\title{CSC 503 Homework Assignment 5}
\author{Due September 22, 2014}
\date{September 15, 2014}

\begin{document}
\maketitle

\begin{enumerate}

\item Use the predicates
  \begin{center}
    \begin{tabular}{rl}
      $F(x,y)$ : & $x$ is the father of $y$ \\
      $M(x,y)$ : & $x$ is the mother of $y$ \\
      $H(x,y)$ : & $x$ is the husband of $y$ \\
      $S(x,y)$ : & $x$ is the sister of $y$ \\
      $B(x,y)$ : & $x$ is the brother of $y$ \\
    \end{tabular}
  \end{center}
and the constant (nullary function) symbols
  \begin{center}
    \begin{tabular}{rl}
      $j$ : & John \\
      $m$ : & Mary \\
    \end{tabular}
  \end{center}
to translate the following English sentences into predicate logic.
You are not allowed to use any predicate, function, or constant
symbols other than the above.
\begin{enumerate}
\item {[4 points]} Everyone has a mother.
\begin{answer}
	\begin{displaymath}
		\forall y \exists x (M(x, y))
	\end{displaymath}
\end{answer}
\item {[4 points]} Everyone has a father and a mother.
\begin{answer}
	\begin{displaymath}
		\forall z \exists x \exists y (F(x, z) \land M(y, z))
	\end{displaymath}
\end{answer}
\item {[4 points]} Whoever has a mother has a father.
\begin{answer}
	\begin{displaymath}
		\forall x (\exists y (M(y, x)) \implies (\exists z (F(z, x)))) 
	\end{displaymath}
\end{answer}
\item {[4 points]} John is a grandfather
\begin{answer}
	\begin{displaymath}
		\exists x,y ((F(x,y) \lor M(x, y)) \land F(j, x))
	\end{displaymath}
\end{answer}
\item {[4 points]} All fathers are parents
\begin{answer}
	\begin{displaymath}
		\forall x,y (F(x,y) \implies (F(x,y) \lor M(x, y)))
	\end{displaymath}
\end{answer}
\item {[4 points]} All husbands are spouses.
\begin{answer}
	\begin{displaymath}
		\forall x, y(H(x, y) \implies (H(x, y) \lor H(y, x)))
	\end{displaymath}
\end{answer}
\item {[4 points]} No uncle is an aunt.
\begin{answer}
	\begin{displaymath}
		\neg \exists x \exists y \exists z (((F(y,x) \land B(z,y)) \lor (M(y,x) \land
		B(z,y)))  \land ((F(y,x) \land S(z,y)) \lor (M(y,x) \land S(z,y))))
	\end{displaymath}
\end{answer}
\item {[4 points]} Nobody's grandmother is anybody's father.
\begin{answer}
	\begin{displaymath}
		\forall x, y, z, w (((M(z, x) \lor F(z, x)) \land M(y, z)) \implies \neg F(y,
		w))
	\end{displaymath}
\end{answer}
\item {[4 points]} If Mary is her own mother, then she is her own grandmother.
\begin{answer}
	\begin{displaymath}
		M(m, m) \implies \exists x ((M(x, m) \lor F(x, m)) \land M(m, x))
	\end{displaymath}
\end{answer}
\item {[4 points]} John's parents are husband and wife.
\begin{answer}
	\begin{displaymath}
		\exists x, y ((F(y, j) \land M(x, j)) \land H(y, x))
	\end{displaymath}
\end{answer}
\end{enumerate}

\item {[20 points]} Using only the basic natural deduction rules, find
  a proof for
  \begin{displaymath}
    \forall x \forall y (P(y) \implies Q(x))
    \turn
    \exists y P(y) \implies \forall x Q(x).
  \end{displaymath}
  
  \begin{answer}
  	\[
  		\begin{nd}
  			\hypo{1} {\forall x \forall y (P(y) \implies Q(x))} \premise{}
  			\open
  				\hypo{2} {\exists y P(y)} \assumption{}
  				\open[x_0]
  					\hypo{3} {} \assumption{}
  					\have{4} {\forall y (P(y) \implies Q(x_0))} \Ae {1}
					\open[y_0] 
						\hypo{5} {P(y_0)} \assumption{}
						\have{6} {P(y_0) \implies Q(x_0)} \Ae {4}
						\have{7} {Q(x_0)} \ie {5, 6}						 
					\close  					  
					\have{8} {Q(x_0)} \Ee {2, 5-7}
  				\close
  				\have{9} {\forall x (Q(x))} \Ai {3-8}
  			\close
  			\have{10} {\exists y P(y) \implies \forall x Q(x)} \ii {2-9}
  		\end{nd}
  	\]
  	
  	Note: 
  	
  	In step $4$ we are eliminating $\forall x$
  	
  	In step $6$ we are eliminating $\forall y$
  	
  	In step $8$ we are eliminating $\exists y$
  	
  	In step $9$ we are introducing $\forall x$
  \end{answer}

\item {[20 points]} Find a proof for
  \begin{displaymath}
    \exists x \forall y (P(x) \lor \neg Q(y))
    \turn
    \forall y \exists x (P(x) \lor \neg Q(y))).
\end{displaymath}

\begin{answer}
	\[
		\begin{nd}
			\hypo{1} {\exists x \forall y (P(x) \lor \neg Q(y))} \premise{}
			\open[y_0]
				\hypo{2} {} \assumption{}
				\open[x_0]
					\hypo{3} {\forall y (P(x_0) \lor \neg Q(y))} \assumption{}
					\have{4} {(P(x_0) \lor \neg Q(y_0))} \Ae {3}
					\have{5} {\exists x (P(x) \lor \neg Q(y_0))} \Ei {4}
				\close
				\have{6} {\exists x (P(x) \lor \neg Q(y_0))} \Ee {1, 3-5}
			\close
			\have{7} {\forall y \exists x (P(x) \lor \neg Q(y))} \Ai {2-6}
		\end{nd}
	\]
	
	Note: 
  	
  	In step $4$ we are eliminating $\forall y$
  	
  	In step $5$ we are introducing $\exists x$
  	
  	In step $6$ we are eliminating $\exists x$
  	
  	In step $7$ we are introducing $\forall y$
\end{answer}
\item {[20 points]} Find a proof for 
  \begin{displaymath}
    \forall x P(a,x,x),
    \forall z \forall y \forall x (P(x,y,z) \implies P(f(f(x)),y,f(z))) 
    \turn
    P(f(f(a)),a,f(a)).
  \end{displaymath}

\begin{answer}
	\[
		\begin{nd}
			\hypo{1} {\forall x P(a,x,x)} \premise{}
			\hypo{2} {\forall z \forall y \forall x (P(x, y, z) \implies
			P(f(f(x)),y,f(z)))} \premise{}
			\have{3} {P(a, a, a)} \Ae {1}
			\have{4} {\forall y \forall x (P(x,y,a) \implies P(f(f(x)),y,f(a)))} \Ae {2}
			\have{5} {\forall x (P(x,a,a) \implies P(f(f(x)),a,f(a)))} \Ae {4}
			\have{6} {P(a,a,a) \implies P(f(f(a)),a,f(a))} \Ae {5}
			\have{7} {P(f(f(a)),a,f(a))} \ie {3, 6}
		\end{nd}
	\]
	
	Note: 
  	
  	In step $3$ we are eliminating $\forall x$
  	
  	In step $4$ we are eliminating $\forall z$
  	
  	In step $5$ we are eliminating $\forall y$
  	
  	In step $6$ we are eliminating $\forall x$
  	
\end{answer}

\end{enumerate}
\end{document}
