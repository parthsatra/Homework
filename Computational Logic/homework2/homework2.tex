\documentclass{article}
% Package and macro definitions for CSC 503
% Originally prepared August 23, 2012 by Jon Doyle

%%% Page dimensions
\setlength{\oddsidemargin}{0in}
\setlength{\evensidemargin}{0in}
\setlength{\topmargin}{0in}
\setlength{\textheight}{9in}
\setlength{\textwidth}{6.5in}
\setlength{\headheight}{0in}
\setlength{\headsep}{0in}
\setlength{\footskip}{0.5in}

%%% Font and symbol definition packages
\usepackage{times} 
\usepackage{helvet} 
\usepackage{alltt}
\usepackage{amsfonts, amsmath}
\usepackage{amssymb}

%%% The modified Sellinger fitch.sty file
\input{fitchhr.sty}

\newcommand{\Z}{\mathbb{Z}}
\newcommand{\Q}{\mathbb{Q}}
\newcommand{\R}{\mathbb{R}}
\newcommand{\N}{\mathbb{N}}
\def\land{\wedge}
\def\lor{\vee}
\def\implies{\rightarrow}
\def\iff{\leftrightarrow}
\def\turn{\vdash}
\def\Cn{\text{Cn}}
\def\Th{\text{Th}}
\def\defeq{\stackrel{\rm def}{=}}

%%% The environment for providing answers to problems
\newenvironment{answer}%
{\par\noindent\textbf{Answer}\par\noindent}%
{}


\title{CSC 503 Homework Assignment 2}
\author{Due September 10, 2014}
\date{August 27, 2014}

\begin{document}
\maketitle

\begin{enumerate}

\item The formulas of propositional logic implicitly assume
  the binding priorities of the logical connectives put forward in
  Convention 1.3.  Make sure that you fully understand those
  conventions by reinserting all omitted parentheses in the following
  abbreviated statements.

  \begin{enumerate}
  \item {[10 points]} $\mathrel{\neg} p \implies \mathrel{\neg} p \lor
    q \implies \mathrel{\neg} r$

	\begin{answer}
		\begin{enumerate}
		\item Highest Precedence to $\neg$.
		
	    	$( \neg p ) \implies ( \neg p) \lor q \implies (\neg r)$

		\item Next precedence is given to $ \lor $ and $\land $ equally.
		
			$( \neg p ) \implies (( \neg p) \lor q) \implies (\neg r)$

		\item Next precedence is given to $\implies$ and it is right associative in
		nature. 
		
			$( \neg p ) \implies ((( \neg p) \lor q) \implies (\neg r))$
		\end{enumerate}
	\end{answer}
	
  \item {[10 points]} $r \implies q \implies p \lor (q \implies
    \mathrel{\neg} p \land r)$
    
    \begin{answer}
		\begin{enumerate}
		\item Highest Precedence to $\neg$.
		
	    	$r \implies q \implies p \lor (q \implies (\neg p) \land r)$

		\item Next precedence is given to $ \lor $ and $\land $ equally.
		
			$r \implies q \implies (p \lor (q \implies ((\neg p) \land r)))$

		\item Next precedence is given to $\implies$ and it is right associative in
		nature. 
		
			$r \implies (q \implies (p \lor (q \implies ((\neg p) \land r))))$
		\end{enumerate}
	\end{answer}
  \end{enumerate}

\item {[10 points]} List all subformulas of the formula $(s \implies r
  \lor l) \land (\neg q \lor r) \implies (\neg (p \implies s) \implies
  r)$.
  
  \begin{answer}
  
  	The subformulas are as follows.
  	\begin{enumerate}
  	  
  	  \item $l$
  	  \item $p$
  	  \item $q$
  	  \item $r$
  	  \item $s$
  	  \item $(r \lor l)$
  	  \item $s \implies (r \lor l)$
  	  \item $\neg q$
  	  \item $(\neg q \lor r)$
  	  \item $(s \implies (r \lor l) \land (\neg q \lor r))$
  	  \item $p \implies s$
  	  \item $\neg (p \implies s)$
  	  \item $(\neg (p \implies s) \implies r)$
  	  \item $((s \implies (r \lor l)) \land (\neg q \lor r)) \implies (\neg (p
  	  \implies s) \implies r)$.
  	  
   	\end{enumerate}
  	
  \end{answer}

\item {[10 points]} A formula is valid iff it computes T for all its
  valuations; it is satisfiable iff it computes T for at least one of
  its valuations.  Is the formula $(p \lor \neg q) \land (q \lor \neg
  r)$ valid?  Is it satisfiable?

	\begin{answer}
	
		The truth table for the formula $(p \lor \neg q) \land (q \lor \neg r)$ is as
		given below
	
		\begin{displaymath}
  			\begin{array}[t]{|c|c|c|c|c|c|c|c|} \hline
    			p & q & r & \neg q & \neg r & p \lor \neg q & q \lor \neg r & (p \lor
    			\neg q) \land (q \lor \neg r) \\ \hline\hline 
    			T & T & T & F & F & T & T & T\\ \hline
    			T & T & F & F & T & T & T & T\\ \hline
    			T & F & T & T & F & T & F & F\\ \hline
  		 		T & F & F & T & T & T & T & T\\ \hline
  		 		F & T & T & F & F & F & T & F\\ \hline
    			F & T & F & F & T & F & T & F\\ \hline
    			F & F & T & T & F & T & F & F\\ \hline
  	 			F & F & F & T & T & T & T & T\\ \hline
  			\end{array}
  		\end{displaymath}
  		
  		From the table we can see that the truth value of the formula does not
  		compute $T$ for all the evaluations, but it does for a few of them and hence
  		the formula is \textbf{Satisfiable}.
  	
	\end{answer}

\item {[20 points]} Does $\models (\neg(p \implies q) \implies (p
  \lor (\neg p \implies q))) \implies p$ hold?  Justify your answer.

	\begin{answer}
		
		\begin{displaymath}
  			\begin{array}[t]{|c|c|c|c|c|c|c|c|} \hline
    			p & q & p \implies q & \neg (p \implies q) & \neg p & \neg p \implies q &
    			p \lor (\neg p \implies q) & \neg (p \implies q) \implies (p \lor (\neg
    			p \implies q)) \\ \hline\hline 
    			T & T & T & F & F & T & T & T \\ \hline 
    			T & F & F & T & F & T & T & T \\ \hline 
    			F & T & T & F & T & T & T & T \\ \hline
  	 			F & F & T & F & T & F & F & T \\ \hline
  			\end{array}
  		\end{displaymath}
  		
  		\begin{center}
  			Table (a)
  		\end{center}
  		
  		\begin{displaymath}
  			\begin{array}[t]{|c|c|c|} \hline
    			p & \neg (p \implies q) \implies (p \lor (\neg p \implies q)) & (\neg(p
    			\implies q) \implies (p \lor (\neg p \implies q))) \implies p \\ \hline\hline 
    			T & T & T\\ \hline 
  	 			F & T & F\\ \hline
  			\end{array}
  		\end{displaymath}
  		
  		\begin{center}
  			Table (b)
  		\end{center}
  		
  		From the above truth tables we can see that the formula $\models (\neg(p \implies q) \implies (p
  		\lor (\neg p \implies q))) \implies p$ depends on the $p$ and hence the
  		formula \textbf{Does Not Hold}.
	
	\end{answer}
	
\item Prove the validity of the following sequents.  Use only the
  basic rules of natural deduction (no derived rules).
  \begin{enumerate}
  \item {[20 points]} $r \implies q, \neg r \implies p, \neg q
    \implies \neg p \turn q$
    
    \begin{answer}
    	\[
    		\begin{nd}
    			\hypo{1} {r \implies q} \premise{}
    			\hypo{2} {\neg r \implies p} \premise{}
    			\hypo{3} {\neg q \implies \neg p} \premise{}
    			\open 
    				\hypo{4a} {\neg q} \assumption{}
    				\have{4b} {\neg p} \ie {3, 4a}
    				\open
    					\hypo{4ci} {\neg r} \assumption{}
    					\have{4cii} {p} \ie {2, 4ci}
    					\have{4ciii} {\bot} \ne {4b, 4cii}
    				\close
    				\have{4d} {\neg \neg r} \ni {4ci - 4ciii}
    				\have{4e} {r} \nne {4d}
    				\have{4f} {q} \ie {1, 4e}
    				\have{4g} {\bot} \ne {4a, 4f}
    			\close
    			\have{5} {\neg \neg q} \ni {4a-4g}
    			\have{6} {q} \nne {5}
    		\end{nd}
    	\]
    \end{answer}
  \item {[20 points]} $p \implies (q \lor r), \neg q, \neg r \turn \neg p$
  
  \begin{answer}
  	\[
  		\begin{nd}
  			\hypo{1} {p \implies (q \lor r)} \premise{}
  			\hypo{2} {\neg q} \premise{}
  			\hypo{3} {\neg r} \premise{}
  			\open 
  				\hypo{4a} {p} \assumption{}
  				\have{4b} {q \lor r} \ie {1, 4a}
  				\open
  					\hypo{4ci} {q} \assumption{}
  					\have{4cii} {\bot} \ne {2, 4ci}
  				\close
  				\open
  					\hypo{4di} {r} \assumption{}
  					\have{4dii} {\bot} \ne {3, 4di}
  				\close
  				\have{4e} {\bot} \oe {4b, 4ci-4cii, 4di-4dii}
  			\close
  			\have{5} {\neg p} \ni {4a-4e}
  		\end{nd}
  	\]
  \end{answer}
  \end{enumerate}

\end{enumerate}
\end{document}
