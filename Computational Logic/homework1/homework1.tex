\documentclass{article}
% Package and macro definitions for CSC 503
% Originally prepared August 23, 2012 by Jon Doyle

%%% Page dimensions
\setlength{\oddsidemargin}{0in}
\setlength{\evensidemargin}{0in}
\setlength{\topmargin}{0in}
\setlength{\textheight}{9in}
\setlength{\textwidth}{6.5in}
\setlength{\headheight}{0in}
\setlength{\headsep}{0in}
\setlength{\footskip}{0.5in}

%%% Font and symbol definition packages
\usepackage{times} 
\usepackage{helvet} 
\usepackage{alltt}
\usepackage{amsfonts, amsmath}
\usepackage{amssymb}

%%% The modified Sellinger fitch.sty file
\input{fitchhr.sty}

\newcommand{\Z}{\mathbb{Z}}
\newcommand{\Q}{\mathbb{Q}}
\newcommand{\R}{\mathbb{R}}
\newcommand{\N}{\mathbb{N}}
\def\land{\wedge}
\def\lor{\vee}
\def\implies{\rightarrow}
\def\iff{\leftrightarrow}
\def\turn{\vdash}
\def\Cn{\text{Cn}}
\def\Th{\text{Th}}
\def\defeq{\stackrel{\rm def}{=}}

%%% The environment for providing answers to problems
\newenvironment{answer}%
{\par\noindent\textbf{Answer}\par\noindent}%
{}


\title{CSC 503 Homework Assignment 1}
\author{Due September 8, 2014}
\date{August 25, 2014}

\begin{document}
\maketitle

\begin{enumerate}

\item Use $\neg$, $\implies$, $\land$, and $\lor$ to express the
  following declarative sentences in propositional logic over atoms
  $p$, $q$, $r$, etc.  First state what your propositional atoms mean
  in self-contained English sentences, then give the translation of
  the sentence.

  \begin{enumerate}

  \item {[10 points]} If Alice flew to San Francisco, then Alice was
    not in Raleigh yesterday.

	\begin{answer}
	
		$p$ : Alice flew to San Francisco.
		
		$q$ : Alice was in Raleigh yesterday.
		
		$p \implies \neg q$	
		
	\end{answer}	

  \item {[10 points]} Either Bob called his father after he became
    sunburned or Bob's mother called him.
    
    \begin{answer}
    	
    	$p$ : Bob Called his father after he became sunburned
    	
    	$q$ : Bob's mother called him (Bob's father).
    	
    	$p \lor q$
    	
    \end{answer}

  \item {[10 points]} Carol's pet zebra is black and white.
  
  	\begin{answer}
  		
  		$p$ : Carol's pet zebra is black
  		
  		$q$ : Carol's pet zebra is white
  		
  		$p \land q$
  		
  	\end{answer}

  \end{enumerate}

\item {[10 points]} Why is the expression $p \lor q \land
  r$ problematic?  Justify your answer.
  
  \begin{answer}
  
  	The given expression is problematic since it can be interpreted in
  	multiple ways and both the interpretations have different truth values as
  	shown below.
  	
  	$1$. $(p \lor q) \land r$
  
  	\begin{displaymath}
  		\begin{array}[t]{|c|c|c|c|c|} \hline
    		p & q & r & (p \lor q) & (p \lor q) \land r \\ \hline\hline
    		T & T & T & T & T \\ \hline
    		T & T & F & T & F \\ \hline
    		T & F & T & T & T \\ \hline
  	 		T & F & F & T & F \\ \hline
  	 		F & T & T & T & T \\ \hline
    		F & T & F & T & F \\ \hline
    		F & F & T & F & F \\ \hline
  	 		F & F & F & F & F \\ \hline
  		\end{array}
  	\end{displaymath}
  	
  	$2$. $p \lor (q \land r)$
  	
  	\begin{displaymath}
  		\begin{array}[t]{|c|c|c|c|c|} \hline
    		p & q & r & (q \land r) & p \lor (q \land r) \\ \hline\hline
    		T & T & T & T & T \\ \hline
    		T & T & F & F & T \\ \hline
    		T & F & T & F & T \\ \hline
  	 		T & F & F & F & T \\ \hline
  	 		F & T & T & T & T \\ \hline
    		F & T & F & F & F \\ \hline
    		F & F & T & F & F \\ \hline
  	 		F & F & F & F & F \\ \hline
  		\end{array}
  	\end{displaymath}
  	
  \end{answer}

\item {[10 points]} Compute the complete truth table of
  the formula $((p \implies p) \implies q) \implies q$.
  
  \begin{answer}
  	
  	Truth table of the formula $((p \implies p) \implies q) \implies q$ is as
  	follows
  	
  	\begin{displaymath}
  		\begin{array}[t]{|c|c|c|c|c|} \hline
    		p & q & p \implies p & (p \implies p) \implies q & 
    		((p \implies p) \implies q) \implies q \\ \hline\hline 
    		T & T & T & T & T \\ \hline 
    		T & F & T & F & T \\ \hline 
    		F & T & T & T & T \\ \hline 
    		F & F & T & F & T \\ \hline
  		\end{array}
  	\end{displaymath}
  	
  	
  \end{answer}

\item {[10 points]} Show that the sequent $(p \lor q) \implies r, \neg
  r \land \neg q \turn \neg p \implies q$ is not valid by finding a
  valuation in which the truth values of the formulas to the left of
  $\turn$ are T and the truth value of the formula to the right of
  $\turn$ is F.

	\begin{answer}

		This sequent can be proved wrong using the natural deduction as shown below
		
		\[
			\begin{nd}
				\hypo{1} {(p \lor q) \implies r} \premise{}
				\hypo{2} {\neg r \land \neg q} \premise{}
				\have{3} {\neg r} \aeone {2}
				\have{4} {\neg q} \aetwo {2}
				\open
					\hypo{5a} {p \lor q} \assumption{}
					\have{5b} {r} \ie {1, 5a}
					\have{5c} {\bot} \ne {3, 5b}
				\close
				\have{6} {\neg (p \lor q)} \ni {5a-5c}
				\open
					\hypo{7a} {p} \assumption{}
					\have{7b} {p \lor q} \oione {7a}
					\have{7c} {\bot} \ne {6, 7b}
				\close
				\have{8} {\neg p} \ni {7a-7c}
				\open
					\hypo{9a} {\neg p \implies q} \assumption{}
					\have{9b} {q} \ie {8, 9a}
					\have{9c} {\bot} \ne {4, 9b}
				\close
				\have{10} {\neg (\neg p \implies q)} \ni {9a-9c}
			\end{nd}
		\]
		
		This shows that when premises $(p \lor q) \implies r, \neg
  r \land \neg q$ are true then $\neg (\neg p \implies q)$ is also true. Hence
  the expected conclusion $\neg p \implies q$ has to be false. Thus for an
  instance, for a set of values of $(p, q, r)$ as $(T, T, T)$, the L.H.S. is of
  the sequent is true and the R.H.S. is false. Also notice that this is only one
  of the sample solution and the solution is not affected for any value of $r$.
  
	\end{answer}

\item Prove the validity of the following sequents.  Use only the
  basic rules of natural deduction (no derived rules).

  \begin{enumerate}
  \item {[20 points]} $\turn (p \lor q) \implies (s \implies ((p \lor
    q) \land s))$
    
    \begin{answer}
    	
    	\[
    		\begin{nd}
    			\open
    				\hypo{1a} {p \lor q} \assumption{}
    				\open
    					\hypo{1ai} {s} \assumption{}
    					\have{1aii} {(p \vee q) \land s} \ai {1a, 1ai}
    				\close	
    				\have{1b} {s \implies ((p \lor q) \land s)} \ii {1ai-1aii}
    			\close
    			\have{2} {(p \lor q) \implies (s \implies ((p \lor q) \land s))} \ii
    			{1a-1b}
    		\end{nd}
    	\]
    \end{answer}
    
  \item {[20 points]} $q \implies \neg p \turn ((\neg p \land q)
    \implies q) \land (q \implies (\neg p \land q)) $
    
    \begin{answer}
    
    	\[
    		\begin{nd}
    			\hypo{1} {q \implies \neg p} \premise{}
    			\open
    				\hypo{2a} {\neg p \land q} \assumption{}
    				\have{2b} {q} \aetwo {2a}
    			\close
    			\have{3} {(\neg p \land q) \implies q} \ii {2a-2b}
    			\open
    				\hypo{4a} {q} \assumption{}
    				\have{4b} {\neg p} \ie {1,4a}
    				\have{4c} {\neg p \land q} \ai {4a,4b}
    			\close
    			\have{5} {q \implies (\neg p \land q)} \ii {4a-4c}
  				\have{6} {((\neg p \land q) \implies q) \land (q \implies (\neg p \land
  				q))} \ai {3,5}
    		\end{nd}
    	\]
    \end{answer}
  \end{enumerate}

\end{enumerate}
\end{document}
