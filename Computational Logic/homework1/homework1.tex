\documentclass{article}
% Package and macro definitions for CSC 503
% Originally prepared August 23, 2012 by Jon Doyle

%%% Page dimensions
\setlength{\oddsidemargin}{0in}
\setlength{\evensidemargin}{0in}
\setlength{\topmargin}{0in}
\setlength{\textheight}{9in}
\setlength{\textwidth}{6.5in}
\setlength{\headheight}{0in}
\setlength{\headsep}{0in}
\setlength{\footskip}{0.5in}

%%% Font and symbol definition packages
\usepackage{times} 
\usepackage{helvet} 
\usepackage{alltt}
\usepackage{amsfonts, amsmath}
\usepackage{amssymb}

%%% The modified Sellinger fitch.sty file
% These are Selinger's fitch.sty macros modified by Jon Doyle to
% conform with the nomenclature of inference rules in the Huth and
% Ryan textbook.  All the license and disclaimers of fitch.sty are
% maintained.  Extensions copyright (C) Jon Doyle, August 23, 2012.

% Macros for Fitch-style natural deduction. 
% Author: Peter Selinger, University of Ottawa
% Created: Jan 14, 2002
% Modified: Feb 8, 2005
% Version: 0.5
% Copyright: (C) 2002-2005 Peter Selinger
% Filename: fitch.sty
% Documentation: fitchdoc.tex
% URL: http://quasar.mathstat.uottawa.ca/~selinger/fitch/

% License:
%
% This program is free software; you can redistribute it and/or modify
% it under the terms of the GNU General Public License as published by
% the Free Software Foundation; either version 2, or (at your option)
% any later version.
%
% This program is distributed in the hope that it will be useful, but
% WITHOUT ANY WARRANTY; without even the implied warranty of
% MERCHANTABILITY or FITNESS FOR A PARTICULAR PURPOSE. See the GNU
% General Public License for more details.
%
% You should have received a copy of the GNU General Public License
% along with this program; if not, write to the Free Software Foundation, 
% Inc., 59 Temple Place, Suite 330, Boston, MA 02111-1307, USA.

% USAGE EXAMPLE:
% 
% The following is a simple example illustrating the usage of this
% package.  For detailed instructions and additional functionality, see
% the user guide, which can be found in the file fitchdoc.tex.
% 
% \[
% \begin{nd}
%   \hypo{1}  {P\vee Q}   
%   \hypo{2}  {\neg Q}                         
%   \open                              
%   \hypo{3a} {P}
%   \have{3b} {P}        \r{3a}
%   \close                   
%   \open
%   \hypo{4a} {Q}
%   \have{4b} {\neg Q}   \r{2}
%   \have{4c} {\bot}     \ne{4a,4b}
%   \have{4d} {P}        \be{4c}
%   \close                             
%   \have{5}  {P}        \oe{1,3a-3b,4a-4d}                 
% \end{nd}
% \]

{\chardef\x=\catcode`\*
\catcode`\*=11
\global\let\nd*astcode\x}
\catcode`\*=11

% References

\newcount\nd*ctr
\def\nd*render{\expandafter\ifx\expandafter\nd*x\nd*base\nd*x\the\nd*ctr\else\nd*base\ifnum\nd*ctr<0\the\nd*ctr\else\ifnum\nd*ctr>0+\the\nd*ctr\fi\fi\fi}
\expandafter\def\csname nd*-\endcsname{}

\def\nd*num#1{\nd*numo{\nd*render}{#1}\global\advance\nd*ctr1}
\def\nd*numopt#1#2{\nd*numo{$#1$}{#2}}
\def\nd*numo#1#2{\edef\x{#1}\mbox{$\x$}\expandafter\global\expandafter\let\csname nd*-#2\endcsname\x}
\def\nd*ref#1{\expandafter\let\expandafter\x\csname nd*-#1\endcsname\ifx\x\relax%
  \errmessage{Undefined natdeduction reference: #1}\else\mbox{$\x$}\fi}
\def\nd*noop{}
\def\nd*set#1#2{\ifx\relax#1\nd*noop\else\global\def\nd*base{#1}\fi\ifx\relax#2\relax\else\global\nd*ctr=#2\fi}
\def\nd*reset{\nd*set{}{1}}
\def\nd*refa#1{\nd*ref{#1}}
\def\nd*aux#1#2{\ifx#2-\nd*refa{#1}--\def\nd*c{\nd*aux{}}%
  \else\ifx#2,\nd*refa{#1}, \def\nd*c{\nd*aux{}}%
  \else\ifx#2;\nd*refa{#1}; \def\nd*c{\nd*aux{}}%
  \else\ifx#2.\nd*refa{#1}. \def\nd*c{\nd*aux{}}%
  \else\ifx#2)\nd*refa{#1})\def\nd*c{\nd*aux{}}%
  \else\ifx#2(\nd*refa{#1}(\def\nd*c{\nd*aux{}}%
  \else\ifx#2\nd*end\nd*refa{#1}\def\nd*c{}%
  \else\def\nd*c{\nd*aux{#1#2}}%
  \fi\fi\fi\fi\fi\fi\fi\nd*c}
\def\ndref#1{\nd*aux{}#1\nd*end}

% Layer A

% define various dimensions (explained in fitchdoc.tex):
\newlength{\nd*dim} 
\newdimen\nd*depthdim
\newdimen\nd*hsep
\newdimen\ndindent
\ndindent=1em
% user command to redefine dimensions
\def\nddim#1#2#3#4#5#6#7#8{\nd*depthdim=#3\relax\nd*hsep=#6\relax%
\def\nd*height{#1}\def\nd*thickness{#8}\def\nd*initheight{#2}%
\def\nd*indent{#5}\def\nd*labelsep{#4}\def\nd*justsep{#7}}
% set initial dimensions
\nddim{4.5ex}{3.5ex}{1.5ex}{1em}{1.6em}{.5em}{2.5em}{.2mm}

\def\nd*v{\rule[-\nd*depthdim]{\nd*thickness}{\nd*height}}
\def\nd*t{\rule[-\nd*depthdim]{0mm}{\nd*height}\rule[-\nd*depthdim]{\nd*thickness}{\nd*initheight}}
\def\nd*i{\hspace{\nd*indent}} 
\def\nd*s{\hspace{\nd*hsep}}
\def\nd*g#1{\nd*f{\makebox[\nd*indent][c]{$#1$}}}
\def\nd*f#1{\raisebox{0pt}[0pt][0pt]{$#1$}}
\def\nd*u#1{\makebox[0pt][l]{\settowidth{\nd*dim}{\nd*f{#1}}%
    \addtolength{\nd*dim}{2\nd*hsep}\hspace{-\nd*hsep}\rule[-\nd*depthdim]{\nd*dim}{\nd*thickness}}\nd*f{#1}}

% Lists

\def\nd*push#1#2{\expandafter\gdef\expandafter#1\expandafter%
  {\expandafter\nd*cons\expandafter{#1}{#2}}}
\def\nd*pop#1{{\def\nd*nil{\gdef#1{\nd*nil}}\def\nd*cons##1##2%
    {\gdef#1{##1}}#1}}
\def\nd*iter#1#2{{\def\nd*nil{}\def\nd*cons##1##2{##1#2{##2}}#1}}
\def\nd*modify#1#2#3{{\def\nd*nil{\gdef#1{\nd*nil}}\def\nd*cons##1##2%
    {\advance#2-1 ##1\advance#2 1 \ifnum#2=1\nd*push#1{#3}\else%
      \nd*push#1{##2}\fi}#1}}

\def\nd*cont#1{{\def\nd*t{\nd*v}\def\nd*v{\nd*v}\def\nd*g##1{\nd*i}%
    \def\nd*i{\nd*i}\def\nd*nil{\gdef#1{\nd*nil}}\def\nd*cons##1##2%
    {##1\expandafter\nd*push\expandafter#1\expandafter{##2}}#1}}

% Layer B

\newcount\nd*n
\def\nd*beginb{\begingroup\nd*reset\gdef\nd*stack{\nd*nil}\nd*push\nd*stack{\nd*t}%
  \begin{array}{l@{\hspace{\nd*labelsep}}l@{\hspace{\nd*justsep}}l}}
\def\nd*resumeb{\begingroup\begin{array}{l@{\hspace{\nd*labelsep}}l@{\hspace{\nd*justsep}}l}}
\def\nd*endb{\end{array}\endgroup}
\def\nd*hypob#1#2{\nd*f{\nd*num{#1}}&\nd*iter\nd*stack\relax\nd*cont\nd*stack\nd*s\nd*u{#2}&}
\def\nd*haveb#1#2{\nd*f{\nd*num{#1}}&\nd*iter\nd*stack\relax\nd*cont\nd*stack\nd*s\nd*f{#2}&}
\def\nd*havecontb#1#2{&\nd*iter\nd*stack\relax\nd*cont\nd*stack\nd*s\nd*f{\hspace{\ndindent}#2}&}
\def\nd*hypocontb#1#2{&\nd*iter\nd*stack\relax\nd*cont\nd*stack\nd*s\nd*u{\hspace{\ndindent}#2}&}

\def\nd*openb{\nd*push\nd*stack{\nd*i}\nd*push\nd*stack{\nd*t}}
\def\nd*closeb{\nd*pop\nd*stack\nd*pop\nd*stack}
\def\nd*guardb#1#2{\nd*n=#1\multiply\nd*n by 2 \nd*modify\nd*stack\nd*n{\nd*g{#2}}}

% Layer C

\def\nd*clr{\gdef\nd*cmd{}\gdef\nd*typ{\relax}}
\def\nd*sto#1#2#3{\gdef\nd*typ{#1}\gdef\nd*byt{}%
  \gdef\nd*cmd{\nd*typ{#2}{#3}\nd*byt\\}}
\def\nd*chtyp{\expandafter\ifx\nd*typ\nd*hypocontb\def\nd*typ{\nd*havecontb}\else\def\nd*typ{\nd*haveb}\fi}
\def\nd*hypoc#1#2{\nd*chtyp\nd*cmd\nd*sto{\nd*hypob}{#1}{#2}}
\def\nd*havec#1#2{\nd*cmd\nd*sto{\nd*haveb}{#1}{#2}}
\def\nd*hypocontc#1{\nd*chtyp\nd*cmd\nd*sto{\nd*hypocontb}{}{#1}}
\def\nd*havecontc#1{\nd*cmd\nd*sto{\nd*havecontb}{}{#1}}
\def\nd*by#1#2{\ifx\nd*x#2\nd*x\gdef\nd*byt{\mbox{#1}}\else\gdef\nd*byt{\mbox{#1, \ndref{#2}}}\fi}

% multi-line macros
\def\nd*mhypoc#1#2{\nd*mhypocA{#1}#2\\\nd*stop\\}
\def\nd*mhypocA#1#2\\{\nd*hypoc{#1}{#2}\nd*mhypocB}
\def\nd*mhypocB#1\\{\ifx\nd*stop#1\else\nd*hypocontc{#1}\expandafter\nd*mhypocB\fi}
\def\nd*mhavec#1#2{\nd*mhavecA{#1}#2\\\nd*stop\\}
\def\nd*mhavecA#1#2\\{\nd*havec{#1}{#2}\nd*mhavecB}
\def\nd*mhavecB#1\\{\ifx\nd*stop#1\else\nd*havecontc{#1}\expandafter\nd*mhavecB\fi}
\def\nd*mhypocontc#1{\nd*mhypocB#1\\\nd*stop\\}
\def\nd*mhavecontc#1{\nd*mhavecB#1\\\nd*stop\\}

\def\nd*beginc{\nd*beginb\nd*clr}
\def\nd*resumec{\nd*resumeb\nd*clr}
\def\nd*endc{\nd*cmd\nd*endb}
\def\nd*openc{\nd*cmd\nd*clr\nd*openb}
\def\nd*closec{\nd*cmd\nd*clr\nd*closeb}
\let\nd*guardc\nd*guardb

% Layer D

% macros with optional arguments spelled-out
\def\nd*hypod[#1][#2]#3[#4]#5{\ifx\relax#4\relax\else\nd*guardb{1}{#4}\fi\nd*mhypoc{#3}{#5}\nd*set{#1}{#2}}
\def\nd*haved[#1][#2]#3[#4]#5{\ifx\relax#4\relax\else\nd*guardb{1}{#4}\fi\nd*mhavec{#3}{#5}\nd*set{#1}{#2}}
\def\nd*havecont#1{\nd*mhavecontc{#1}}
\def\nd*hypocont#1{\nd*mhypocontc{#1}}
\def\nd*base{undefined}
\def\nd*opend[#1]#2{\nd*cmd\nd*clr\nd*openb\nd*guard{#1}#2}
\def\nd*close{\nd*cmd\nd*clr\nd*closeb}
\def\nd*guardd[#1]#2{\nd*guardb{#1}{#2}}

% Handling of optional arguments.

\def\nd*optarg#1#2#3{\ifx[#3\def\nd*c{#2#3}\else\def\nd*c{#2[#1]{#3}}\fi\nd*c}
\def\nd*optargg#1#2#3{\ifx[#3\def\nd*c{#1#3}\else\def\nd*c{#2{#3}}\fi\nd*c}

\def\nd*five#1{\nd*optargg{\nd*four{#1}}{\nd*two{#1}}}
\def\nd*four#1[#2]{\nd*optarg{0}{\nd*three{#1}[#2]}}
\def\nd*three#1[#2][#3]#4{\nd*optarg{}{#1[#2][#3]{#4}}}
\def\nd*two#1{\nd*three{#1}[\relax][]}

\def\nd*have{\nd*five{\nd*haved}}
\def\nd*hypo{\nd*five{\nd*hypod}}
\def\nd*open{\nd*optarg{}{\nd*opend}}
\def\nd*guard{\nd*optarg{1}{\nd*guardd}}

\def\nd*init{%
  \let\open\nd*open%
  \let\close\nd*close%
  \let\hypo\nd*hypo%
  \let\have\nd*have%
  \let\hypocont\nd*hypocont%
  \let\havecont\nd*havecont%
  \let\by\nd*by%
  \let\guard\nd*guard%
  \def\ii{\by{$\rightarrow$i}}%    %JD modification
  \def\ie{\by{$\rightarrow$e}}%    %JD modification
  \def\bi{\by{$\leftrightarrow$i}}%    %JD modification
  \def\be{\by{$\leftrightarrow$e}}%    %JD modification
  \def\Ai{\by{$\forall$i}}%
  \def\Ae{\by{$\forall$e}}%
  \def\Ei{\by{$\exists$i}}%
  \def\Ee{\by{$\exists$e}}%
  \def\ai{\by{$\wedge$i}}%
  \def\ae{\by{$\wedge$e}}%
  \def\aeone{\by{$\wedge$e$_1$}}%    %JD addition
  \def\aetwo{\by{$\wedge$e$_2$}}%    %JD addition
  \def\ai{\by{$\wedge$i}}%
  \def\aione{\by{$\wedge$i$_1$}}%    %JD addition for false proof questions
  \def\aitwo{\by{$\wedge$i$_2$}}%    %JD addition for false proof questions
  \def\ae{\by{$\wedge$e}}%
  \def\oi{\by{$\vee$i}}%
  \def\oione{\by{$\vee$i$_1$}}%    %JD addition
  \def\oitwo{\by{$\vee$i$_2$}}%    %JD addition
  \def\oe{\by{$\vee$e}}%
  \def\oeone{\by{$\vee$e$_1$}}%    %JD addition for false proof questions
  \def\oetwo{\by{$\vee$e$_2$}}%    %JD addition for false proof questions
  \def\ni{\by{$\neg$i}}%
  \def\ne{\by{$\neg$e}}%
  \def\be{\by{$\bot$e}}%
  \def\nne{\by{$\neg\neg$e}}%JD addition
  \def\nni{\by{$\neg\neg$i}}%JD addition
  \def\r{\by{R}}%
  \def\copy{\by{copy}}% %JD addition
  \def\pbc{\by{PBC}}%    %JD addition
  \def\mt{\by{MT}}%    %JD addition
  \def\lem{\by{LEM}}%    %JD addition
  \def\bn{\by{$\bot_\neg$}}%
  \def\tn{\by{$\top_\neg$}}%
  \def\premise{\by{premise}} % JD addition
  \def\assumption{\by{assumption}} % JD addition
  \def\Implied{\by{Implied}} % JD addition
  \def\Assignment{\by{Assignment}} % JD addition
  \def\IfStatement{\by{If-Statement}} % JD addition
  \def\PartialWhile{\by{Partial-While}} % JD addition
  \def\InvariantGuard{\by{Invariant Hyp. and Guard}} % JD addition
  \def\TotalWhile{\by{Total-While}} % JD addition
  \def\Ki{\by{$K$i}}%
  \def\Ke{\by{$K$e}}%
  \def\KT{\by{$KT$}}%
  \def\Kfour{\by{$K4$}}%
  \def\Kfive{\by{$K5$}}%
  \def\EKi{\by{$E$i}}%
  \def\EKe{\by{$E$e}}%
  \def\CKi{\by{$C$i}}%
  \def\CKe{\by{$C$e}}%
  \def\CK{\by{$CK$}}%
  \def\Cfour{\by{$C4$}}%
  \def\Cfive{\by{$C5$}}%
}

\newenvironment{nd}{\begingroup\nd*init\nd*beginc}{\nd*endc\endgroup}
\newenvironment{ndresume}{\begingroup\nd*init\nd*resumec}{\nd*endc\endgroup}

\catcode`\*=\nd*astcode

% End of file fitch.sty



\newcommand{\Z}{\mathbb{Z}}
\newcommand{\Q}{\mathbb{Q}}
\newcommand{\R}{\mathbb{R}}
\newcommand{\N}{\mathbb{N}}
\def\land{\wedge}
\def\lor{\vee}
\def\implies{\rightarrow}
\def\iff{\leftrightarrow}
\def\turn{\vdash}
\def\Cn{\text{Cn}}
\def\Th{\text{Th}}
\def\defeq{\stackrel{\rm def}{=}}

%%% The environment for providing answers to problems
\newenvironment{answer}%
{\par\noindent\textbf{Answer}\par\noindent}%
{}


\title{CSC 503 Homework Assignment 1}
\author{Due September 8, 2014}
\date{August 25, 2014}

\begin{document}
\maketitle

\begin{enumerate}

\item Use $\neg$, $\implies$, $\land$, and $\lor$ to express the
  following declarative sentences in propositional logic over atoms
  $p$, $q$, $r$, etc.  First state what your propositional atoms mean
  in self-contained English sentences, then give the translation of
  the sentence.

  \begin{enumerate}

  \item {[10 points]} If Alice flew to San Francisco, then Alice was
    not in Raleigh yesterday.

	\begin{answer}
	
		$p$ : Alice flew to San Francisco.
		
		$q$ : Alice was in Raleigh yesterday.
		
		$p \implies \neg q$	
		
	\end{answer}	

  \item {[10 points]} Either Bob called his father after he became
    sunburned or Bob's mother called him.
    
    \begin{answer}
    	
    	$p$ : Bob Called his father after he became sunburned
    	
    	$q$ : Bob's mother called him (Bob's father).
    	
    	$p \lor q$
    	
    \end{answer}

  \item {[10 points]} Carol's pet zebra is black and white.
  
  	\begin{answer}
  		
  		$p$ : Carol's pet zebra is black
  		
  		$q$ : Carol's pet zebra is white
  		
  		$p \land q$
  		
  	\end{answer}

  \end{enumerate}

\item {[10 points]} Why is the expression $p \lor q \land
  r$ problematic?  Justify your answer.
  
  \begin{answer}
  
  	The given expression is problematic since it can be interpreted in
  	multiple ways and both the interpretations have different truth values as
  	shown below.
  	
  	$1$. $(p \lor q) \land r$
  
  	\begin{displaymath}
  		\begin{array}[t]{|c|c|c|c|c|} \hline
    		p & q & r & (p \lor q) & (p \lor q) \land r \\ \hline\hline
    		T & T & T & T & T \\ \hline
    		T & T & F & T & F \\ \hline
    		T & F & T & T & T \\ \hline
  	 		T & F & F & T & F \\ \hline
  	 		F & T & T & T & T \\ \hline
    		F & T & F & T & F \\ \hline
    		F & F & T & F & F \\ \hline
  	 		F & F & F & F & F \\ \hline
  		\end{array}
  	\end{displaymath}
  	
  	$2$. $p \lor (q \land r)$
  	
  	\begin{displaymath}
  		\begin{array}[t]{|c|c|c|c|c|} \hline
    		p & q & r & (q \land r) & p \lor (q \land r) \\ \hline\hline
    		T & T & T & T & T \\ \hline
    		T & T & F & F & T \\ \hline
    		T & F & T & F & T \\ \hline
  	 		T & F & F & F & T \\ \hline
  	 		F & T & T & T & T \\ \hline
    		F & T & F & F & F \\ \hline
    		F & F & T & F & F \\ \hline
  	 		F & F & F & F & F \\ \hline
  		\end{array}
  	\end{displaymath}
  	
  \end{answer}

\item {[10 points]} Compute the complete truth table of
  the formula $((p \implies p) \implies q) \implies q$.
  
  \begin{answer}
  	
  	Truth table of the formula $((p \implies p) \implies q) \implies q$ is as
  	follows
  	
  	\begin{displaymath}
  		\begin{array}[t]{|c|c|c|c|c|} \hline
    		p & q & p \implies p & (p \implies p) \implies q & 
    		((p \implies p) \implies q) \implies q \\ \hline\hline 
    		T & T & T & T & T \\ \hline 
    		T & F & T & F & T \\ \hline 
    		F & T & T & T & T \\ \hline 
    		F & F & T & F & T \\ \hline
  		\end{array}
  	\end{displaymath}
  	
  	
  \end{answer}

\item {[10 points]} Show that the sequent $(p \lor q) \implies r, \neg
  r \land \neg q \turn \neg p \implies q$ is not valid by finding a
  valuation in which the truth values of the formulas to the left of
  $\turn$ are T and the truth value of the formula to the right of
  $\turn$ is F.

	\begin{answer}

		This sequent can be proved wrong using the natural deduction as shown below
		
		\[
			\begin{nd}
				\hypo{1} {(p \lor q) \implies r} \premise{}
				\hypo{2} {\neg r \land \neg q} \premise{}
				\have{3} {\neg r} \aeone {2}
				\have{4} {\neg q} \aetwo {2}
				\open
					\hypo{5a} {p \lor q} \assumption{}
					\have{5b} {r} \ie {1, 5a}
					\have{5c} {\bot} \ne {3, 5b}
				\close
				\have{6} {\neg (p \lor q)} \ni {5a-5c}
				\open
					\hypo{7a} {p} \assumption{}
					\have{7b} {p \lor q} \oione {7a}
					\have{7c} {\bot} \ne {6, 7b}
				\close
				\have{8} {\neg p} \ni {7a-7c}
				\open
					\hypo{9a} {\neg p \implies q} \assumption{}
					\have{9b} {q} \ie {8, 9a}
					\have{9c} {\bot} \ne {4, 9b}
				\close
				\have{10} {\neg (\neg p \implies q)} \ni {9a-9c}
			\end{nd}
		\]
		
		This shows that when premises $(p \lor q) \implies r, \neg
  r \land \neg q$ are true then $\neg (\neg p \implies q)$ is also true. Hence
  the expected conclusion $\neg p \implies q$ has to be false. Thus for an
  instance, for a set of values of $(p, q, r)$ as $(T, T, T)$, the L.H.S. is of
  the sequent is true and the R.H.S. is false. Also notice that this is only one
  of the sample solution and the solution is not affected for any value of $r$.
  
	\end{answer}

\item Prove the validity of the following sequents.  Use only the
  basic rules of natural deduction (no derived rules).

  \begin{enumerate}
  \item {[20 points]} $\turn (p \lor q) \implies (s \implies ((p \lor
    q) \land s))$
    
    \begin{answer}
    	
    	\[
    		\begin{nd}
    			\open
    				\hypo{1a} {p \lor q} \assumption{}
    				\open
    					\hypo{1ai} {s} \assumption{}
    					\have{1aii} {(p \vee q) \land s} \ai {1a, 1ai}
    				\close	
    				\have{1b} {s \implies ((p \lor q) \land s)} \ii {1ai-1aii}
    			\close
    			\have{2} {(p \lor q) \implies (s \implies ((p \lor q) \land s))} \ii
    			{1a-1b}
    		\end{nd}
    	\]
    \end{answer}
    
  \item {[20 points]} $q \implies \neg p \turn ((\neg p \land q)
    \implies q) \land (q \implies (\neg p \land q)) $
    
    \begin{answer}
    
    	\[
    		\begin{nd}
    			\hypo{1} {q \implies \neg p} \premise{}
    			\open
    				\hypo{2a} {\neg p \land q} \assumption{}
    				\have{2b} {q} \aetwo {2a}
    			\close
    			\have{3} {(\neg p \land q) \implies q} \ii {2a-2b}
    			\open
    				\hypo{4a} {q} \assumption{}
    				\have{4b} {\neg p} \ie {1,4a}
    				\have{4c} {\neg p \land q} \ai {4a,4b}
    			\close
    			\have{5} {q \implies (\neg p \land q)} \ii {4a-4c}
  				\have{6} {((\neg p \land q) \implies q) \land (q \implies (\neg p \land
  				q))} \ai {3,5}
    		\end{nd}
    	\]
    \end{answer}
  \end{enumerate}

\end{enumerate}
\end{document}
