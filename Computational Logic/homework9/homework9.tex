\documentclass{article}
% Package and macro definitions for CSC 503
% Originally prepared August 23, 2012 by Jon Doyle

%%% Page dimensions
\setlength{\oddsidemargin}{0in}
\setlength{\evensidemargin}{0in}
\setlength{\topmargin}{0in}
\setlength{\textheight}{9in}
\setlength{\textwidth}{6.5in}
\setlength{\headheight}{0in}
\setlength{\headsep}{0in}
\setlength{\footskip}{0.5in}

%%% Font and symbol definition packages
\usepackage{times} 
\usepackage{helvet} 
\usepackage{alltt}
\usepackage{amsfonts, amsmath}
\usepackage{amssymb}

%%% The modified Sellinger fitch.sty file
% These are Selinger's fitch.sty macros modified by Jon Doyle to
% conform with the nomenclature of inference rules in the Huth and
% Ryan textbook.  All the license and disclaimers of fitch.sty are
% maintained.  Extensions copyright (C) Jon Doyle, August 23, 2012.

% Macros for Fitch-style natural deduction. 
% Author: Peter Selinger, University of Ottawa
% Created: Jan 14, 2002
% Modified: Feb 8, 2005
% Version: 0.5
% Copyright: (C) 2002-2005 Peter Selinger
% Filename: fitch.sty
% Documentation: fitchdoc.tex
% URL: http://quasar.mathstat.uottawa.ca/~selinger/fitch/

% License:
%
% This program is free software; you can redistribute it and/or modify
% it under the terms of the GNU General Public License as published by
% the Free Software Foundation; either version 2, or (at your option)
% any later version.
%
% This program is distributed in the hope that it will be useful, but
% WITHOUT ANY WARRANTY; without even the implied warranty of
% MERCHANTABILITY or FITNESS FOR A PARTICULAR PURPOSE. See the GNU
% General Public License for more details.
%
% You should have received a copy of the GNU General Public License
% along with this program; if not, write to the Free Software Foundation, 
% Inc., 59 Temple Place, Suite 330, Boston, MA 02111-1307, USA.

% USAGE EXAMPLE:
% 
% The following is a simple example illustrating the usage of this
% package.  For detailed instructions and additional functionality, see
% the user guide, which can be found in the file fitchdoc.tex.
% 
% \[
% \begin{nd}
%   \hypo{1}  {P\vee Q}   
%   \hypo{2}  {\neg Q}                         
%   \open                              
%   \hypo{3a} {P}
%   \have{3b} {P}        \r{3a}
%   \close                   
%   \open
%   \hypo{4a} {Q}
%   \have{4b} {\neg Q}   \r{2}
%   \have{4c} {\bot}     \ne{4a,4b}
%   \have{4d} {P}        \be{4c}
%   \close                             
%   \have{5}  {P}        \oe{1,3a-3b,4a-4d}                 
% \end{nd}
% \]

{\chardef\x=\catcode`\*
\catcode`\*=11
\global\let\nd*astcode\x}
\catcode`\*=11

% References

\newcount\nd*ctr
\def\nd*render{\expandafter\ifx\expandafter\nd*x\nd*base\nd*x\the\nd*ctr\else\nd*base\ifnum\nd*ctr<0\the\nd*ctr\else\ifnum\nd*ctr>0+\the\nd*ctr\fi\fi\fi}
\expandafter\def\csname nd*-\endcsname{}

\def\nd*num#1{\nd*numo{\nd*render}{#1}\global\advance\nd*ctr1}
\def\nd*numopt#1#2{\nd*numo{$#1$}{#2}}
\def\nd*numo#1#2{\edef\x{#1}\mbox{$\x$}\expandafter\global\expandafter\let\csname nd*-#2\endcsname\x}
\def\nd*ref#1{\expandafter\let\expandafter\x\csname nd*-#1\endcsname\ifx\x\relax%
  \errmessage{Undefined natdeduction reference: #1}\else\mbox{$\x$}\fi}
\def\nd*noop{}
\def\nd*set#1#2{\ifx\relax#1\nd*noop\else\global\def\nd*base{#1}\fi\ifx\relax#2\relax\else\global\nd*ctr=#2\fi}
\def\nd*reset{\nd*set{}{1}}
\def\nd*refa#1{\nd*ref{#1}}
\def\nd*aux#1#2{\ifx#2-\nd*refa{#1}--\def\nd*c{\nd*aux{}}%
  \else\ifx#2,\nd*refa{#1}, \def\nd*c{\nd*aux{}}%
  \else\ifx#2;\nd*refa{#1}; \def\nd*c{\nd*aux{}}%
  \else\ifx#2.\nd*refa{#1}. \def\nd*c{\nd*aux{}}%
  \else\ifx#2)\nd*refa{#1})\def\nd*c{\nd*aux{}}%
  \else\ifx#2(\nd*refa{#1}(\def\nd*c{\nd*aux{}}%
  \else\ifx#2\nd*end\nd*refa{#1}\def\nd*c{}%
  \else\def\nd*c{\nd*aux{#1#2}}%
  \fi\fi\fi\fi\fi\fi\fi\nd*c}
\def\ndref#1{\nd*aux{}#1\nd*end}

% Layer A

% define various dimensions (explained in fitchdoc.tex):
\newlength{\nd*dim} 
\newdimen\nd*depthdim
\newdimen\nd*hsep
\newdimen\ndindent
\ndindent=1em
% user command to redefine dimensions
\def\nddim#1#2#3#4#5#6#7#8{\nd*depthdim=#3\relax\nd*hsep=#6\relax%
\def\nd*height{#1}\def\nd*thickness{#8}\def\nd*initheight{#2}%
\def\nd*indent{#5}\def\nd*labelsep{#4}\def\nd*justsep{#7}}
% set initial dimensions
\nddim{4.5ex}{3.5ex}{1.5ex}{1em}{1.6em}{.5em}{2.5em}{.2mm}

\def\nd*v{\rule[-\nd*depthdim]{\nd*thickness}{\nd*height}}
\def\nd*t{\rule[-\nd*depthdim]{0mm}{\nd*height}\rule[-\nd*depthdim]{\nd*thickness}{\nd*initheight}}
\def\nd*i{\hspace{\nd*indent}} 
\def\nd*s{\hspace{\nd*hsep}}
\def\nd*g#1{\nd*f{\makebox[\nd*indent][c]{$#1$}}}
\def\nd*f#1{\raisebox{0pt}[0pt][0pt]{$#1$}}
\def\nd*u#1{\makebox[0pt][l]{\settowidth{\nd*dim}{\nd*f{#1}}%
    \addtolength{\nd*dim}{2\nd*hsep}\hspace{-\nd*hsep}\rule[-\nd*depthdim]{\nd*dim}{\nd*thickness}}\nd*f{#1}}

% Lists

\def\nd*push#1#2{\expandafter\gdef\expandafter#1\expandafter%
  {\expandafter\nd*cons\expandafter{#1}{#2}}}
\def\nd*pop#1{{\def\nd*nil{\gdef#1{\nd*nil}}\def\nd*cons##1##2%
    {\gdef#1{##1}}#1}}
\def\nd*iter#1#2{{\def\nd*nil{}\def\nd*cons##1##2{##1#2{##2}}#1}}
\def\nd*modify#1#2#3{{\def\nd*nil{\gdef#1{\nd*nil}}\def\nd*cons##1##2%
    {\advance#2-1 ##1\advance#2 1 \ifnum#2=1\nd*push#1{#3}\else%
      \nd*push#1{##2}\fi}#1}}

\def\nd*cont#1{{\def\nd*t{\nd*v}\def\nd*v{\nd*v}\def\nd*g##1{\nd*i}%
    \def\nd*i{\nd*i}\def\nd*nil{\gdef#1{\nd*nil}}\def\nd*cons##1##2%
    {##1\expandafter\nd*push\expandafter#1\expandafter{##2}}#1}}

% Layer B

\newcount\nd*n
\def\nd*beginb{\begingroup\nd*reset\gdef\nd*stack{\nd*nil}\nd*push\nd*stack{\nd*t}%
  \begin{array}{l@{\hspace{\nd*labelsep}}l@{\hspace{\nd*justsep}}l}}
\def\nd*resumeb{\begingroup\begin{array}{l@{\hspace{\nd*labelsep}}l@{\hspace{\nd*justsep}}l}}
\def\nd*endb{\end{array}\endgroup}
\def\nd*hypob#1#2{\nd*f{\nd*num{#1}}&\nd*iter\nd*stack\relax\nd*cont\nd*stack\nd*s\nd*u{#2}&}
\def\nd*haveb#1#2{\nd*f{\nd*num{#1}}&\nd*iter\nd*stack\relax\nd*cont\nd*stack\nd*s\nd*f{#2}&}
\def\nd*havecontb#1#2{&\nd*iter\nd*stack\relax\nd*cont\nd*stack\nd*s\nd*f{\hspace{\ndindent}#2}&}
\def\nd*hypocontb#1#2{&\nd*iter\nd*stack\relax\nd*cont\nd*stack\nd*s\nd*u{\hspace{\ndindent}#2}&}

\def\nd*openb{\nd*push\nd*stack{\nd*i}\nd*push\nd*stack{\nd*t}}
\def\nd*closeb{\nd*pop\nd*stack\nd*pop\nd*stack}
\def\nd*guardb#1#2{\nd*n=#1\multiply\nd*n by 2 \nd*modify\nd*stack\nd*n{\nd*g{#2}}}

% Layer C

\def\nd*clr{\gdef\nd*cmd{}\gdef\nd*typ{\relax}}
\def\nd*sto#1#2#3{\gdef\nd*typ{#1}\gdef\nd*byt{}%
  \gdef\nd*cmd{\nd*typ{#2}{#3}\nd*byt\\}}
\def\nd*chtyp{\expandafter\ifx\nd*typ\nd*hypocontb\def\nd*typ{\nd*havecontb}\else\def\nd*typ{\nd*haveb}\fi}
\def\nd*hypoc#1#2{\nd*chtyp\nd*cmd\nd*sto{\nd*hypob}{#1}{#2}}
\def\nd*havec#1#2{\nd*cmd\nd*sto{\nd*haveb}{#1}{#2}}
\def\nd*hypocontc#1{\nd*chtyp\nd*cmd\nd*sto{\nd*hypocontb}{}{#1}}
\def\nd*havecontc#1{\nd*cmd\nd*sto{\nd*havecontb}{}{#1}}
\def\nd*by#1#2{\ifx\nd*x#2\nd*x\gdef\nd*byt{\mbox{#1}}\else\gdef\nd*byt{\mbox{#1, \ndref{#2}}}\fi}

% multi-line macros
\def\nd*mhypoc#1#2{\nd*mhypocA{#1}#2\\\nd*stop\\}
\def\nd*mhypocA#1#2\\{\nd*hypoc{#1}{#2}\nd*mhypocB}
\def\nd*mhypocB#1\\{\ifx\nd*stop#1\else\nd*hypocontc{#1}\expandafter\nd*mhypocB\fi}
\def\nd*mhavec#1#2{\nd*mhavecA{#1}#2\\\nd*stop\\}
\def\nd*mhavecA#1#2\\{\nd*havec{#1}{#2}\nd*mhavecB}
\def\nd*mhavecB#1\\{\ifx\nd*stop#1\else\nd*havecontc{#1}\expandafter\nd*mhavecB\fi}
\def\nd*mhypocontc#1{\nd*mhypocB#1\\\nd*stop\\}
\def\nd*mhavecontc#1{\nd*mhavecB#1\\\nd*stop\\}

\def\nd*beginc{\nd*beginb\nd*clr}
\def\nd*resumec{\nd*resumeb\nd*clr}
\def\nd*endc{\nd*cmd\nd*endb}
\def\nd*openc{\nd*cmd\nd*clr\nd*openb}
\def\nd*closec{\nd*cmd\nd*clr\nd*closeb}
\let\nd*guardc\nd*guardb

% Layer D

% macros with optional arguments spelled-out
\def\nd*hypod[#1][#2]#3[#4]#5{\ifx\relax#4\relax\else\nd*guardb{1}{#4}\fi\nd*mhypoc{#3}{#5}\nd*set{#1}{#2}}
\def\nd*haved[#1][#2]#3[#4]#5{\ifx\relax#4\relax\else\nd*guardb{1}{#4}\fi\nd*mhavec{#3}{#5}\nd*set{#1}{#2}}
\def\nd*havecont#1{\nd*mhavecontc{#1}}
\def\nd*hypocont#1{\nd*mhypocontc{#1}}
\def\nd*base{undefined}
\def\nd*opend[#1]#2{\nd*cmd\nd*clr\nd*openb\nd*guard{#1}#2}
\def\nd*close{\nd*cmd\nd*clr\nd*closeb}
\def\nd*guardd[#1]#2{\nd*guardb{#1}{#2}}

% Handling of optional arguments.

\def\nd*optarg#1#2#3{\ifx[#3\def\nd*c{#2#3}\else\def\nd*c{#2[#1]{#3}}\fi\nd*c}
\def\nd*optargg#1#2#3{\ifx[#3\def\nd*c{#1#3}\else\def\nd*c{#2{#3}}\fi\nd*c}

\def\nd*five#1{\nd*optargg{\nd*four{#1}}{\nd*two{#1}}}
\def\nd*four#1[#2]{\nd*optarg{0}{\nd*three{#1}[#2]}}
\def\nd*three#1[#2][#3]#4{\nd*optarg{}{#1[#2][#3]{#4}}}
\def\nd*two#1{\nd*three{#1}[\relax][]}

\def\nd*have{\nd*five{\nd*haved}}
\def\nd*hypo{\nd*five{\nd*hypod}}
\def\nd*open{\nd*optarg{}{\nd*opend}}
\def\nd*guard{\nd*optarg{1}{\nd*guardd}}

\def\nd*init{%
  \let\open\nd*open%
  \let\close\nd*close%
  \let\hypo\nd*hypo%
  \let\have\nd*have%
  \let\hypocont\nd*hypocont%
  \let\havecont\nd*havecont%
  \let\by\nd*by%
  \let\guard\nd*guard%
  \def\ii{\by{$\rightarrow$i}}%    %JD modification
  \def\ie{\by{$\rightarrow$e}}%    %JD modification
  \def\bi{\by{$\leftrightarrow$i}}%    %JD modification
  \def\be{\by{$\leftrightarrow$e}}%    %JD modification
  \def\Ai{\by{$\forall$i}}%
  \def\Ae{\by{$\forall$e}}%
  \def\Ei{\by{$\exists$i}}%
  \def\Ee{\by{$\exists$e}}%
  \def\ai{\by{$\wedge$i}}%
  \def\ae{\by{$\wedge$e}}%
  \def\aeone{\by{$\wedge$e$_1$}}%    %JD addition
  \def\aetwo{\by{$\wedge$e$_2$}}%    %JD addition
  \def\ai{\by{$\wedge$i}}%
  \def\aione{\by{$\wedge$i$_1$}}%    %JD addition for false proof questions
  \def\aitwo{\by{$\wedge$i$_2$}}%    %JD addition for false proof questions
  \def\ae{\by{$\wedge$e}}%
  \def\oi{\by{$\vee$i}}%
  \def\oione{\by{$\vee$i$_1$}}%    %JD addition
  \def\oitwo{\by{$\vee$i$_2$}}%    %JD addition
  \def\oe{\by{$\vee$e}}%
  \def\oeone{\by{$\vee$e$_1$}}%    %JD addition for false proof questions
  \def\oetwo{\by{$\vee$e$_2$}}%    %JD addition for false proof questions
  \def\ni{\by{$\neg$i}}%
  \def\ne{\by{$\neg$e}}%
  \def\be{\by{$\bot$e}}%
  \def\nne{\by{$\neg\neg$e}}%JD addition
  \def\nni{\by{$\neg\neg$i}}%JD addition
  \def\r{\by{R}}%
  \def\copy{\by{copy}}% %JD addition
  \def\pbc{\by{PBC}}%    %JD addition
  \def\mt{\by{MT}}%    %JD addition
  \def\lem{\by{LEM}}%    %JD addition
  \def\bn{\by{$\bot_\neg$}}%
  \def\tn{\by{$\top_\neg$}}%
  \def\premise{\by{premise}} % JD addition
  \def\assumption{\by{assumption}} % JD addition
  \def\Implied{\by{Implied}} % JD addition
  \def\Assignment{\by{Assignment}} % JD addition
  \def\IfStatement{\by{If-Statement}} % JD addition
  \def\PartialWhile{\by{Partial-While}} % JD addition
  \def\InvariantGuard{\by{Invariant Hyp. and Guard}} % JD addition
  \def\TotalWhile{\by{Total-While}} % JD addition
  \def\Ki{\by{$K$i}}%
  \def\Ke{\by{$K$e}}%
  \def\KT{\by{$KT$}}%
  \def\Kfour{\by{$K4$}}%
  \def\Kfive{\by{$K5$}}%
  \def\EKi{\by{$E$i}}%
  \def\EKe{\by{$E$e}}%
  \def\CKi{\by{$C$i}}%
  \def\CKe{\by{$C$e}}%
  \def\CK{\by{$CK$}}%
  \def\Cfour{\by{$C4$}}%
  \def\Cfive{\by{$C5$}}%
}

\newenvironment{nd}{\begingroup\nd*init\nd*beginc}{\nd*endc\endgroup}
\newenvironment{ndresume}{\begingroup\nd*init\nd*resumec}{\nd*endc\endgroup}

\catcode`\*=\nd*astcode

% End of file fitch.sty



\newcommand{\Z}{\mathbb{Z}}
\newcommand{\Q}{\mathbb{Q}}
\newcommand{\R}{\mathbb{R}}
\newcommand{\N}{\mathbb{N}}
\def\land{\wedge}
\def\lor{\vee}
\def\implies{\rightarrow}
\def\iff{\leftrightarrow}
\def\turn{\vdash}
\def\Cn{\text{Cn}}
\def\Th{\text{Th}}
\def\defeq{\stackrel{\rm def}{=}}

%%% The environment for providing answers to problems
\newenvironment{answer}%
{\par\noindent\textbf{Answer}\par\noindent}%
{}


\usepackage{amsfonts, amsmath, amsthm}
\usepackage{tikz}
\usetikzlibrary{arrows,automata}

\def\Sometime{\mathord{\mathsf{F}}}
\def\Forever{\mathord{\mathsf{G}}}
\def\Next{\mathord{\mathsf{O}}}
\def\NextX{\mathord{\mathsf{X}}}
\def\Until{\mathrel{\mathsf{U}}}
\def\Release{\mathrel{\mathsf{R}}}
\def\WeakUntil{\mathrel{\mathsf{W}}}
\def\Before{\mathrel{\mathsf{B}}}
\def\True{\mathord{\mathsf{true}}}
\def\All{\mathord{\mathsf{A}}}
\def\Exists{\mathord{\mathsf{E}}}
\def\Every{\mathord{\mathsf{E}}}

\title{CSC 503 Homework Assignment 9}
\author{Due October 29, 2014}
\date{October 22, 2014}

\begin{document}
\maketitle

\begin{itemize}
\item \textbf{[40 points total]} Consider the transition model ${\cal
    M}_2$ depicted in Figure \ref{f2} in answering the following
  questions about LTL and CTL statements.
  \begin{figure}[h]
    \centering
    \caption{Model ${\cal M}_2$}
\begin{center}

\begin{tikzpicture}[>=stealth',shorten >=1pt,auto,node distance=3cm]
  \node[state] (q3)      {$s_3: q,r$};
  \node[state] (q4)    [right of=q3] {$s_4: p,q$};
  \node[initial,state] (q1)   [below of=q3]   {$s_1: p,r,t$};
  \node[state] (q2)   [right of=q1]   {$s_2: r$};

  \path[->] 
        (q3) edge         node {} (q4)
        (q4) edge [loop right] node {} (q4)
        (q1) edge         node {} (q3)
        (q2) edge         node {} (q4)
        (q2) edge [loop right] node {} (q2)
        (q1) edge    [bend left] node {} (q2)
        (q2) edge    [bend left] node {} (q1);
\end{tikzpicture}
\end{center}
\label{f2}
\end{figure}
  \begin{enumerate}
  \item \textbf{[8 points]} Does ${\cal M}_2, s_4 \models r \implies
    q$?  Explain why or why not.
    \begin{answer}
    	${\cal M}_2, s_4 \models r \implies q$ is True. \newline
    	\textbf{Explanation}: In the state $s_4$, $r$ is false and then the
    	implication is always true. So according to the definition this holds.
    \end{answer}
    \bigskip
  \item \textbf{[8 points]} Does ${\cal M}_2, s_1 \models r \implies
    q$?  Explain why or why not.
    \begin{answer}
    	${\cal M}_2, s_1 \models r \implies q$ is False. \newline
    	\textbf{Explanation}: In the state $s_1$ in the model, $r$ is true. Thus
    	for the implication to be true $q$ also needs to be true in state $s_1$,
    	which is not the case. Hence this does not hold.
    \end{answer}
    \bigskip
  \item \textbf{[8 points]} Does ${\cal M}_2, s_3 \models \Sometime
    \neg t$?  Explain why or why not.
    \begin{answer}
    	${\cal M}_2, s_3 \models \Sometime \neg t$ is True. \newline
    	\textbf{Explanation}: The Model starting from state $s_3$ contains only one
    	path which is $s_3-(s_4)$. In both the state $neg t$ is true since $t$ is
    	false. Thus $\Sometime \neg t$ is true since it occurs atleast once in the
    	path.
    \end{answer}
    \newpage
  \item \textbf{[8 points]} Does ${\cal M}_2, s_1 \models \Sometime
    \neg t$?  Explain why or why not.
    \begin{answer}
    	${\cal M}_2, s_1 \models \Sometime \neg t$ is True. \newline
    	\textbf{Explanation}: $t$ is only true in state $s_1$. Hence $\neg t$ is
    	false in state $s_1$ only. All the paths starting at state $s_1$ have
    	atleast one transition to states $s_2$ or $s_3$ and thus atleast once
    	$\neg t$ is true. Thus overall for all paths from state $s_1$ $\Sometime
    	\neg t$ is also true.
    \end{answer}
    \bigskip
  \item \textbf{[8 points]} Does ${\cal M}_2, s_2 \models \neg \Exists
    \Forever r$?  Explain why or why not.
    \begin{answer}
    	${\cal M}_2, s_2 \models \neg \Exists \Forever r$ is False. \newline
    	\textbf{Explanation}: The formula means that there exists no path where
    	$\Forever r$ is true. Now consider a path $(s_2)$, which is looping through
    	$s_2$ till infinity. In this path $r$ is always true. Thus $\Forever r$ is
    	also true. And since we found atleast one path starting from $s_2$ where
    	$r$ is always true, so $\Exists \Forever r$ is also true, which means that
    	the given expression is False since its a complementary formula to
    	$\Exists \Forever r$.
    \end{answer}
    \bigskip
  \item \textbf{[8 points]} Does ${\cal M}_2, s_4 \models \neg \Exists
    \Forever r$?  Explain why or why not.
    \begin{answer}
    	${\cal M}_2, s_4 \models \neg \Exists \Forever r$ is True.	\newline
    	\textbf{Explanation}: The model contains only a single path from starting
    	from $s_4$ which is $(s_4)$. The expression $\neg \Exists \Forever r$ can
    	also be represented as $\All \Sometime \neg r$. The first formula means
    	that there doesn't exists any path for which $r$ is always true and if we
    	take the negative sign inside then we get the second formula which means
    	for all paths $\neg r$ exists sometime in the future. Now in the path
    	$(s_4)$ which is the only one in the model starting with state $s_4$ has
    	$\neg r$ as true since $r$ is false. Thus $\Sometime \neg r$ is also true
    	and since its the only path so $\All \Sometime \neg r$ is also true. Thus
    	$\neg \Exists \Forever r$ is also true since both are the same.
    \end{answer}
    \bigskip
  \item \textbf{[8 points]} Does ${\cal M}_2, s_3 \models \Exists
    \Sometime q$?  Explain why or why not.
    \begin{answer}
    	${\cal M}_2, s_3 \models \Exists \Sometime q$ is True.	\newline
    	\textbf{Explanation}: Model starting from $s_3$ has only one path which is
    	$s_3-(s_4)$. In this path we can see that $q$ is true in state $s_3$ as
    	well as $s_4$ on the path. Thus $\Exists \Sometime q$ is true.
    \end{answer}
    \bigskip
  \item \textbf{[8 points]} Does ${\cal M}_2, s_1 \models \Exists
    \Sometime q$?  Explain why or why not.
    \begin{answer}
    	${\cal M}_2, s_1 \models \Exists \Sometime q$ is True.	\newline	
    	\textbf{Explanation}: The formula $\Exists \Sometime q$ means that there
    	exists atleast one path where $\Sometime q$ is true. Consider a path
    	$s_1-s_3-(s_4)$ which is one of the path in the Model. In this path $q$ is
    	true in state $s_3$ as well as $s_4$. Thus $\Sometime q$ is also true. And
    	since we found one path where $\Sometime q$ is true, thus $\Exists
    	\Sometime q$ is also true.
    \end{answer}
    \bigskip
  \item \textbf{[8 points]} Does ${\cal M}_2, s_4 \models \All
    \NextX \All \Forever (p \lor q)$?  Explain why or why not.
    \begin{answer}
    	${\cal M}_2, s_4 \models \All \NextX \All \Forever (p \lor q)$ is True.
    	\newline
    	\textbf{Explanation}: The formula $\All \NextX \All \Forever (p \lor q)$
    	means that the formula $\All \Forever (p \lor q)$ is true next on all
    	paths. In the model starting from state $s_4$ there is only one path
    	$(s_4)$. On this path $(p \lor q)$ is always true starting from the second
    	node which is again $s_4$. Thus $\All \NextX \All \Forever (p \lor q)$ is
    	true as there is only one path and the formula holds in that path.
    \end{answer}
    \bigskip
  \item \textbf{[8 points]} Does ${\cal M}_2, s_2 \models \All \NextX
    \All \Forever (p \lor q)$?  Explain why or why not.
    \begin{answer}
    	${\cal M}_2, s_2 \models \All \NextX \All \Forever (p \lor q)$ is False.
    	\newline
    	\textbf{Explanation}: The formula $\All \NextX \All \Forever (p \lor q)$
    	means that the formula $\All \Forever (p \lor q)$ is true next on all
    	paths. Consider one of the paths as $(s_2)$, which is looping through state
    	$s_2$. In this path neither $p$ nor $q$ is ever true. Thus we know that for
    	this particular path violates the formula. Hence $\All \NextX \All
    	\Forever (p \lor q)$ is false.
     \end{answer}
    \bigskip
  \end{enumerate}

%\newpage

\item \textbf{[20 points]} List all subformulas of the CTL formula,
  using Convention 3.13 (p.\ 209) to resolve ambiguities.
  \begin{displaymath}
    \Exists [\All \NextX p \Until (\neg \Exists \NextX q \land \All
    [\neg p \Until \All \Forever (p \lor q)])]
  \end{displaymath}
\end{itemize}
	\begin{answer}
	The subformula of the given formula is as follows:-
		\begin{enumerate}
		  \item $\Exists [\All \NextX p \Until (\neg \Exists \NextX q \land \All [\neg
		  p \Until \All \Forever (p \lor q)])]$
		  \item $\All \NextX p$
		  \item $p$
		  \item $\neg \Exists \NextX q \land \All [\neg p \Until \All \Forever (p \lor
		  q)]$
		  \item $\neg \Exists \NextX q$
		  \item $\Exists \NextX q$
		  \item $q$
		  \item $\All [\neg p \Until \All \Forever (p \lor q)]$
		  \item $\neg p$
		  \item $\All \Forever (p \lor q)$
		  \item $p \lor q$
		\end{enumerate}
	\end{answer}
\end{document}
