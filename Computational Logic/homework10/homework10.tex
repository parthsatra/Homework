\documentclass{article}
% Package and macro definitions for CSC 503
% Originally prepared August 23, 2012 by Jon Doyle

%%% Page dimensions
\setlength{\oddsidemargin}{0in}
\setlength{\evensidemargin}{0in}
\setlength{\topmargin}{0in}
\setlength{\textheight}{9in}
\setlength{\textwidth}{6.5in}
\setlength{\headheight}{0in}
\setlength{\headsep}{0in}
\setlength{\footskip}{0.5in}

%%% Font and symbol definition packages
\usepackage{times} 
\usepackage{helvet} 
\usepackage{alltt}
\usepackage{amsfonts, amsmath}
\usepackage{amssymb}

%%% The modified Sellinger fitch.sty file
\input{fitchhr.sty}

\newcommand{\Z}{\mathbb{Z}}
\newcommand{\Q}{\mathbb{Q}}
\newcommand{\R}{\mathbb{R}}
\newcommand{\N}{\mathbb{N}}
\def\land{\wedge}
\def\lor{\vee}
\def\implies{\rightarrow}
\def\iff{\leftrightarrow}
\def\turn{\vdash}
\def\Cn{\text{Cn}}
\def\Th{\text{Th}}
\def\defeq{\stackrel{\rm def}{=}}

%%% The environment for providing answers to problems
\newenvironment{answer}%
{\par\noindent\textbf{Answer}\par\noindent}%
{}

\usepackage{stmaryrd}

\def\Sometime{\mathord{\mathsf{F}}}
\def\Forever{\mathord{\mathsf{G}}}
\def\Next{\mathord{\mathsf{O}}}
\def\NextX{\mathord{\mathsf{X}}}
\def\Until{\mathrel{\mathsf{U}}}
\def\Release{\mathrel{\mathsf{R}}}
\def\WeakUntil{\mathrel{\mathsf{W}}}
\def\Before{\mathrel{\mathsf{B}}}
\def\True{\mathord{\mathsf{true}}}
\def\All{\mathord{\mathsf{A}}}
\def\Exists{\mathord{\mathsf{E}}}
\def\Every{\mathord{\mathsf{E}}}

\def\True{\mathord{\mathtt{true}}}
\def\False{\mathord{\mathtt{false}}}
\def\If{\mathrel{\mathtt{if}}}
\def\Else{\mathrel{\mathtt{else}}}
\def\While{\mathrel{\mathtt{while}}}
\def\IfElse#1#2#3{\If #1 \ \{ #2 \} \Else \{ #3 \}}
\def\Whiledo#1#2{\While #1\ \{ #2 \}}
\def\Hcond#1{\llparenthesis #1 \rrparenthesis}
\def\Hoare#1#2#3{\Hcond{#1} \mathrel{#2} \Hcond{#3}}

\def\parmodels{\mathrel{\models_{\textup{par}}}}
\def\totmodels{\mathrel{\models_{\textup{tot}}}}
\def\parturn{\mathrel{\turn_{\textup{par}}}}
\def\totturn{\mathrel{\turn_{\textup{tot}}}}

\title{CSC 503 Homework Assignment 10}
\author{Due November 10, 2014}
\date{November 3, 2014}

\begin{document}
\maketitle

In the following proofs about programs, use the natural deduction
proof environment, the \verb+\Hcond+ macro, and the Hoare-logic
justifications \verb+\Implied{}+, \verb+\Assignment{}+,
\verb+\IfStatement{}+, \verb+\PartialWhile{}+, \\
\verb+\InvariantGuard{}+, and \verb+\TotalWhile{}+ to present proofs
of partial or total correctness.  If you are not using LaTeX, you can
substitute double parentheses for the Hoare condition delimiters
$\Hcond{~}$.  For example, we can rewrite the proof in Example 4.13.1
on page 277 of the textbook as
  \begin{displaymath}
    \begin{nd}
      \have{} {\Hcond{y = 5}}
      \have{} {\Hcond{y + 1 = 6}}         \Implied{}
      \have{} {\texttt{\textbf{x = y + 1;}}}   
      \have{} {\Hcond{x = 6}}          \Assignment{}
    \end{nd}
  \end{displaymath}
  The standard stmaryrd package is also used in these macros.

\begin{enumerate}

\item {[40 points]} Use the proof rule for assignment and arithmetic
  entailment as appropriate to show the validity of
  \begin{displaymath}
    \parturn \Hoare{\top}{P}{y = 2x}.
  \end{displaymath}
where $P$ is the program 
  \begin{displaymath}
    \begin{nd}
      \have{} {\texttt{\textbf{y = z;}}}   
      \have{} {\texttt{\textbf{y = x + x;}}}          
    \end{nd}
  \end{displaymath}
  \begin{answer}
  	\begin{displaymath}
    \begin{nd}
      \have{} {\Hcond{\top}}
      \have{} {\Hcond{x + x = 2x}} \Implied{}
      \have{} {\texttt{\textbf{y = z;}}}   
      \have{} {\Hcond{x + x = 2x}} \Assignment{}
      \have{} {\texttt{\textbf{y = x + x;}}}
      \have{} {\Hcond{y = 2x}} \Assignment{}          
    \end{nd}
  \end{displaymath}
  \end{answer}

\newpage
\item {[60 points]} Prove the validity of the sequent $\parturn
  \Hoare{\top}{Q}{z = \min(x,y)}$ where $\min(x,y)$ is the smallest
  number of $x$ and $y$ and the code of $Q$ is given by
  \begin{displaymath}
    \begin{nd}
      \have{} {\texttt{\textbf{if (x < y) \{}}}   
      \have{} {\quad\texttt{\textbf{z = x;}}}          
      \have{} {\texttt{\textbf{\} else \{}}}                   
      \have{} {\quad\texttt{\textbf{z = y;}}}   
      \have{} {\texttt{\textbf{\}}}}                  
    \end{nd}
  \end{displaymath}
  \begin{answer}
  \begin{displaymath}
    \begin{nd}
      \have{} {\Hcond{\top}}
      \have{} {((x < y) \implies x = min(x,y)) \land (\neg(x < y) \implies y =
      min(x,y))} \Implied{}
      \have{} {\texttt{\textbf{if (x < y) \{}}}   
      \have{} {\Hcond{x = \min(x,y)}} \IfStatement{}
      \have{} {\quad\texttt{\textbf{z = x;}}}     
      \have{} {\Hcond{z = \min(x,y)}} \Assignment{}     
      \have{} {\texttt{\textbf{\} else \{}}}     
      \have{} {\Hcond{y = \min(x,y)}} \IfStatement{}              
      \have{} {\quad\texttt{\textbf{z = y;}}}
      \have{} {\Hcond{z = \min(x,y)}} \Assignment{}   
      \have{} {\texttt{\textbf{\}}}}   
      \have{} {\Hcond{z = \min(x,y)}} \IfStatement{}               
    \end{nd}
  \end{displaymath}
  \end{answer}
\end{enumerate}
\end{document}
