\documentclass{article}
% Package and macro definitions for CSC 503
% Originally prepared August 23, 2012 by Jon Doyle

%%% Page dimensions
\setlength{\oddsidemargin}{0in}
\setlength{\evensidemargin}{0in}
\setlength{\topmargin}{0in}
\setlength{\textheight}{9in}
\setlength{\textwidth}{6.5in}
\setlength{\headheight}{0in}
\setlength{\headsep}{0in}
\setlength{\footskip}{0.5in}

%%% Font and symbol definition packages
\usepackage{times} 
\usepackage{helvet} 
\usepackage{alltt}
\usepackage{amsfonts, amsmath}
\usepackage{amssymb}

%%% The modified Sellinger fitch.sty file
\input{fitchhr.sty}

\newcommand{\Z}{\mathbb{Z}}
\newcommand{\Q}{\mathbb{Q}}
\newcommand{\R}{\mathbb{R}}
\newcommand{\N}{\mathbb{N}}
\def\land{\wedge}
\def\lor{\vee}
\def\implies{\rightarrow}
\def\iff{\leftrightarrow}
\def\turn{\vdash}
\def\Cn{\text{Cn}}
\def\Th{\text{Th}}
\def\defeq{\stackrel{\rm def}{=}}

%%% The environment for providing answers to problems
\newenvironment{answer}%
{\par\noindent\textbf{Answer}\par\noindent}%
{}


\title{CSC 503 Homework Assignment 3}
\author{Due September 15, 2014}
\date{September 8, 2014}

\begin{document}
\maketitle

\begin{enumerate}

\item {[20 points]} Construct a formula $\phi$ in \textbf{DNF} based on
  the following truth table:
  \begin{center}
    \begin{tabular}{cc|c}
      $p$ & $q$ & $\phi$ \\ \hline
      T & T & F \\
      T & F & T \\
      F & T & T \\
      F & F & T \\
    \end{tabular}
  \end{center}
  \begin{answer}
  	From the above truth tables we can derive the DNF with following steps
  	\begin{enumerate}
  	  \item Rows with truth values '$T$' are $2, 3, 4$.
  	  \item The formula $\phi$ can be represented in DNF as $(Row2 \lor Row3 \lor
  	  Row4)$ which means that except for these there can be no other values as $T$.
  	  \item Each Row can be represented by the corresponding values of the
  	  literals. So the formula becomes $\phi : (p \land \neg q) \lor (\neg p
  	  \land q) \lor (\neg p \land \neg q)$
	\end{enumerate}
	
	Thus the DNF for the given truth table is $(p \land \neg q) \lor (\neg p
  	  \land q) \lor (\neg p \land \neg q)$
  \end{answer}

\item {[20 points]} Construct a formula $\phi$ in \textbf{CNF} based on
  the following truth table:
  \begin{center}
    \begin{tabular}{ccc|c}
      $p$ & $q$ & $r$ & $\phi$ \\ \hline
      T & T & T & T\\
      T & T & F & T\\
      T & F & T & F\\
      T & F & F & F\\
      F & T & T & T\\
      F & T & F & F\\
      F & F & T & F\\
      F & F & F & T\\
    \end{tabular}
  \end{center}
  \begin{answer}
  	\begin{enumerate}
  		\item Rows with truth values $F$ are $3, 4, 6, 7$
  		\item The formula $\phi$ can be represented in CNF as $(\neg Row3 \land
  		\neg Row4 \land \neg Row6 \land \neg Row7)$ which means that except for the
  		false value of the rows $3, 4, 6, 7$ others are $T$.
  		\item Each Row can be represented by the corresponding values of the
  		literals. So the formula becomes $\phi : (\neg (p \land \neg q \land r)
  		\land \neg(p \land \neg q \land \neg r) \land \neg(\neg p \land q \land
  		\neg r) \land \neg(\neg p \land \neg q \land r))$
  		\item Reducing the above formula using De'Morgans Law we get $\phi: ((\neg
  		p \lor q \lor \neg r) \land (\neg p \lor q \lor r) \land (p \lor \neg q
  		\lor r) \land (p \lor q \lor \neg r))$
  	\end{enumerate}
  	
  	Thus the CNF for the given truth table is $((\neg
  		p \lor q \lor \neg r) \land (\neg p \lor q \lor r) \land (p \lor \neg q
  		\lor r) \land (p \lor q \lor \neg r))$
  \end{answer}

  \item {[40 points]} Apply algorithm HORN from page 66 of the textbook
    to the following Horn formula.
  \begin{displaymath}
    \begin{array}{ll}
    (p \land q \land w \implies \bot) &\land \\
    (t \implies \bot) &\land \\
    (r \implies p) &\land \\
    (\top \implies r) &\land \\
    (\top \implies q) &\land \\
    (r \land u \implies w) &\land \\
    (u \implies s) &\land \\
    (\top \implies u)
    \end{array}
  \end{displaymath}
Your answer should list propositional letters in the order in which
they are marked as well as giving the overall answer.

\begin{answer}
	\begin{enumerate}
	  \item Mark all occurances of $\top$.
	  	
	  \item Mark $r, q, u$ from $(\top \implies r), (\top \implies q),
	  (\top \implies u)$
	  
		\begin{displaymath}
    		\begin{array}{ll}
   				(p \land \textbf{q} \land w \implies \bot) &\land \\
    			(t \implies \bot) &\land \\
    			(\textbf{r} \implies p) &\land \\
    			(\top \implies \textbf{r}) &\land \\
    			(\top \implies \textbf{q}) &\land \\
    			(\textbf{r} \land \textbf{u} \implies w) &\land \\
    			(\textbf{u} \implies s) &\land \\
    			(\top \implies \textbf{u})
    		\end{array}
  		\end{displaymath}
  		\item Mark $p, w, s$ from $(r \implies p), (r \land u \implies w), (u
  		\implies s)$
  		
		\begin{displaymath}
    		\begin{array}{ll}
   				(\textbf{p} \land \textbf{q} \land \textbf{w} \implies \bot) &\land \\
    			(t \implies \bot) &\land \\
    			(\textbf{r} \implies \textbf{p}) &\land \\
    			(\top \implies \textbf{r}) &\land \\
    			(\top \implies \textbf{q}) &\land \\
    			(\textbf{r} \land \textbf{u} \implies \textbf{w}) &\land \\
    			(\textbf{u} \implies \textbf{s}) &\land \\
    			(\top \implies \textbf{u})
    		\end{array}
  		\end{displaymath}
  		
  		\item Mark $\bot$ from $(\textbf{p} \land \textbf{q} \land \textbf{w}
  		\implies \bot)$
  		
	\end{enumerate}
	
	The order order of propositional letters is $r, q, u, p, w, s, \bot$. 
	Since we have marked $\bot$, the horn formula is \textbf{'Unsatisfiable'}.
\end{answer}

\item {[20 points]}  Explain in what sense the SAT solving
  technique, as presented in lecture and the book, can be used to
  check whether formulas are tautologies.
  
  \begin{answer}
  	SAT solving technique is an extension of the HORN Marking algorithm where we
  	can compute the contraints on the subformulas of the general formula $\phi$
  	that makes $\phi$ true.
  	
  	There are multiple SAT solving techniques. One that solves in linear time of
  	the number to nodes in a directed acyclic graph(DAG) of the formula $phi$ and
  	the other one which is an enhancement to this which solves in cubic time.
  	
  	\textbf{Linear Solver}
  	We use the marking algorithm on the parse tree of the formula. Before running
  	the algorithm we transform the formula into the semantically equivalent form
  	having same propositional atoms. The algorithm to check whether formulas are
  	as follows:
  	
  	\begin{enumerate}
  	  \item Take the formula $\phi$ we want to check for tautology.
  	  \item Convert the formula $\phi$ using the following transformations
  	  	\begin{enumerate}
  	  	  \item $T(p) = p$
  	  	  \item $T(\neg \phi) = \neg T(\phi)$
  	  	  \item $T(\phi_{1} \land \phi_{2}) = T(\phi_{1}) \land T(\phi_{2})$
  	  	  \item $T(\phi_{1} \lor \phi_{2}) = \neg(\neg T(\phi_{1}) \land \neg
  	  	  T(\phi_{2}))$
  	  	  \item $T(\phi_{1} \implies \phi_{2}) = \neg (T(\phi_{1}) \land \neg
  	  	  T(\phi_{2}))$
  	  	\end{enumerate}
  	  \item Now we can only prove if the formula is tautology if we can prove
  	  that its negation is never satisfiable. So consider the $\neg \phi$ as the
  	  new formula and we would like to see if that is satisfiable.
  	  \item Construct the parse tree for the formula $\neg \phi$
  	  \item Consider the root node of the tree to be true and then apply the
  	  following rules to find out the corresponding contraints of the child nodes.
  	  \begin{enumerate}
  	    \item Negation Laws: If the current node is $\neg$ and has the value $p$
  	    then child node should have the constraint $\neg p$ i.e. if the current
  	    node has constraint $T$ then the child node will have $\neg T = F$ and
  	    vice versa.
  	    \item Conjunction Laws: In this case if the current node is $T$ then
  	    each child node will have a constraint $T$. But if the current node is
  	    $F$ then atleast one of the child must have $F$. So anyone of the child
  	    can be false. 
  	  \end{enumerate}
  	  \item Continue doing the expansion till we reach the leaf nodes or reach a
  	  contradiction. If we reach the contradiction then we say that the formula
  	  is not satisfiable. Which means $\neg \phi$ is not satisfiable and thus
  	  $\phi$ must be a tautology. 
  	  However if the leaf node is reached then we go bottom-up tp verify whether
  	  all the contraints in conjunction hold good and the root node is still in
  	  $T$ state. If this is the case then we say that $\neg \phi$ is satisfiable
  	  and hence $\phi$ is not a tautology.
  	  
  	\end{enumerate}
  	
  	\textbf{Note:} In the above algorithm we might reach a state where we have
  	  not reached even the contradiction nor the leaf node. In this case we will
  	  have some free (not constrained) nodes. In such cases assume temperory
  	  values for these nodes. Say first time we assume $T$ for the node and we
  	  reach a contradiction them we assume the permanent value of that node as
  	  $\neg T = F$ and re- compute the constraint. If we reach a contradiction
  	  again then we can conclude that $\neg \phi$ is not satisfiable. Otherwise
  	  it is satisfiable and $\phi$ is not a tautology. This extension of the
  	  first algorithm takes cubic time to compute and hence is called a
  	  \textbf{cubic solver}.
  \end{answer}

\end{enumerate}
\end{document}
