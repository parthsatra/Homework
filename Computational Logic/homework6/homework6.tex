\documentclass{article}
% Package and macro definitions for CSC 503
% Originally prepared August 23, 2012 by Jon Doyle

%%% Page dimensions
\setlength{\oddsidemargin}{0in}
\setlength{\evensidemargin}{0in}
\setlength{\topmargin}{0in}
\setlength{\textheight}{9in}
\setlength{\textwidth}{6.5in}
\setlength{\headheight}{0in}
\setlength{\headsep}{0in}
\setlength{\footskip}{0.5in}

%%% Font and symbol definition packages
\usepackage{times} 
\usepackage{helvet} 
\usepackage{alltt}
\usepackage{amsfonts, amsmath}
\usepackage{amssymb}

%%% The modified Sellinger fitch.sty file
\input{fitchhr.sty}

\newcommand{\Z}{\mathbb{Z}}
\newcommand{\Q}{\mathbb{Q}}
\newcommand{\R}{\mathbb{R}}
\newcommand{\N}{\mathbb{N}}
\def\land{\wedge}
\def\lor{\vee}
\def\implies{\rightarrow}
\def\iff{\leftrightarrow}
\def\turn{\vdash}
\def\Cn{\text{Cn}}
\def\Th{\text{Th}}
\def\defeq{\stackrel{\rm def}{=}}

%%% The environment for providing answers to problems
\newenvironment{answer}%
{\par\noindent\textbf{Answer}\par\noindent}%
{}


\title{CSC 503 Homework Assignment 6}
\author{Due October 1, 2014}
\date{September 24, 2014}

\begin{document}
\maketitle

\noindent
The algorithm presented in lecture to calculate the most general
unifier of a set $S$ consists of the following steps.
\begin{itemize}
\item Step 0:
  \begin{itemize}
  \item Set $S_0 = S$
  \item Set $\sigma_0 = \epsilon$
  \end{itemize}

\item Step $k+1$:
    \begin{itemize}
    \item If $|S_k| = 1$, return $\sigma_0\cdots\sigma_k$
    \item If the disagreement set $D(S_k)$ contains both a variable
      $v$ and a term $t$ in which $v$ \emph{does not occur}, then
      \begin{itemize}
      \item Choose least such pair
      \item Set $\sigma_{k+1} = \{ t/v \}$
      \item Set $S_{k+1} = S_k \sigma_{k+1}$
      \item Proceed to step $k+2$
      \end{itemize}
    \item Otherwise, announce that $S$ has no unifier
    \end{itemize}
  \end{itemize}

\begin{enumerate}

\item Apply the unification algorithm to each of the following sets.
  For each set, at each step $i$, show (a) the disagreement of $S_i$,
  (b) the substitution $\sigma_i$ if there is one, or an explanation
  why there is no unifying substitution, (c) the result $S_{i+1}$ of
  applying $\sigma_i$ to $S_i$.  If the set unifies, show also (d) the
  overall substitution $\sigma_0 \dots \sigma_k$ expressed as a single
  substitution, not as a composition, and (e) the formula resulting
  from applying the most general unifier to the expressions in the
  set.

  In the following expressions, assume that $a,b,c$ are constant
  symbols, $f,g,h$ are function symbols, $P,Q$ are predicate symbols,
  and $u,v,w,x,y,z$ are variable symbols.

  \begin{enumerate}
  \item {[25 points]}
    $S = \{ P(f(x),y), P(y,f(z)) \}$
    \begin{answer}
		Initializing $\sigma_0$ to $\{\}$
		
		$S_0 = \{ P(f(x),y), P(y,f(z)) \}$
		
		$D(S_0) = \{f(x), y\}$
		\bigskip
		
		$\sigma_1 = \{f(x)/y\}$
		
		$S_1 = \{ P(f(x),f(x)), P(f(x),f(z)) \}$
		
		$D(S_1) = \{x, z\}$
		\bigskip
		
		$\sigma_2 = \{x/z\}$

		$S_2 = \{ P(f(x), f(x))\}$

		$|S_2| = 1$		
		\bigskip
		
		$\sigma = \sigma_0 \cdot{} \sigma_1 \cdot{} \sigma_2$
		
		$\sigma = \{\} \cdot{} \{f(x)/y\} \cdot{} \{x/z\}$
		
		$\sigma = \{f(x)/y\} \cdot{} \{x/z\}$
		
		$\sigma = \{f(x)/y, x/z\}$
		
		Unification is feasible for above $\sigma$.
    \end{answer}
    
  \item {[25 points]}
    $S = \{ P(f(x),f(f(y))), P(f(y),f(g(z))) \}$
    
    \begin{answer}
    	Initializing $\sigma_0$ to $\{\}$
    	
    	$S_0 = \{ P(f(x),f(f(y))), P(f(y),f(g(z))) \}$
    	
    	$D(S_0) = \{ x, y\}$
    	\bigskip
    	
    	$\sigma_1 = \{x/y\}$
    	
    	$S_1 = \{ P(f(x),f(f(x))), P(f(x),f(g(z))) \}$
    	
    	$D(S_1) = \{ f(x), g(z)\}$
    	
    	There is no substitution for making 'f' and 'g' equal and hence we cannot
    	unify these formulas.
    	\bigskip
    	
    	Unification is not feasible.
    \end{answer}

  \item {[25 points]}
    $S = \{ P(x,f(f(x))), P(y,y) \}$

	\begin{answer}
		Initializing $\sigma_0$ to $\{\}$
		
		$S_0 = \{ P(x,f(f(x))), P(y,y) \}$
		
		$D(S_0) = \{x, y\}$
		\bigskip
		
		$\sigma_1 = \{x/y\}$
		
		$S_1 = \{ P(x,f(f(x))), P(x,x) \}$
		
		$D(S_1) = \{x, f(f(x))\}$
		
		No substitution is possible for this as we cannot substitute $f(f(x))$ for $x$
		as $f(f(x))$ contains $x$.
		\bigskip
		
		Unification is not feasible.
		
	\end{answer}
  \item {[25 points]}
    $S = \{ Q(f(g(v),a),h(w,b)), Q(f(x,y),h(w,w)), Q(f(g(v),a),h(v,b)) \}$
    
    \begin{answer}
    	Initializing $\sigma_0$ to $\{\}$
    	
    	$S_0 = \{ Q(f(g(v),a),h(w,b)), Q(f(x,y),h(w,w)), Q(f(g(v),a),h(v,b)) \}$
    	
    	$D(S_0) = \{ g(v), x\}$
    	\bigskip
    	
    	$\sigma_1 = \{ g(v)/x \}$
    	
    	$S_1 = \{ Q(f(g(v),a),h(w,b)), Q(f(g(v),y),h(w,w)), Q(f(g(v),a),h(v,b)) \}$
    	
    	$D(S_1) = \{ a, y\}$
    	\bigskip
    	
    	$\sigma_2 = \{ a/y \}$
    	
    	$S_2 = \{ Q(f(g(v),a),h(w,b)), Q(f(g(v),a),h(w,w)), Q(f(g(v),a),h(v,b)) \}$
    	
    	$D(S_2) = \{ w, v\}$
    	\bigskip
    	
    	$\sigma_3 = \{ v/w \}$
    	
    	$S_3 = \{ Q(f(g(v),a),h(v,b)), Q(f(g(v),a),h(v,v))) \}$
    	
    	$D(S_3) = \{ v, b\}$
    	\bigskip
    	
    	$\sigma_5 = \{ b/v \}$
    	
    	$S_5 = \{ Q(f(g(b),a),h(b,b))) \}$
    	
    	$|S_5| = 1$
    	\bigskip
    	
    	$\sigma = \sigma_0 \cdot{} \sigma_1 \cdot{} \sigma_2 \cdot{} \sigma_3
    	\cdot{} \sigma_4$
		
		$\sigma = \{\} \cdot{} \{g(v)/x\} \cdot{} \{ a/y \} \cdot{} \{v/w \} \cdot{}
		\{ b/v \}$
		
		$\sigma = \{g(v)/x\} \cdot{} \{ a/y \} \cdot{} \{v/w \} \cdot{} \{ b/v \}$
		
		$\sigma = \{g(v)/x, a/y \} \cdot{} \{v/w \} \cdot{} \{ b/v \}$
		
		$\sigma = \{g(v)/x, a/y, v/w \} \cdot{} \{ b/v \}$
		
		$\sigma = \{g(b)/x, a/y, b/w, b/v \}$
		\bigskip
		
		Unification is feasible using the above $\sigma$.	    	
    \end{answer}
  \end{enumerate}

\end{enumerate}
\end{document}
